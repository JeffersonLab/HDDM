%% ****** Start of file template.aps ****** %
%%
%%
%%   This file is part of the APS files in the REVTeX 4 distribution.
%%   Version 4.0 of REVTeX, August 2001
%%
%%
%%   Copyright (c) 2001 The American Physical Society.
%%
%%   See the REVTeX 4 README file for restrictions and more information.
%%
%
% This is a template for producing manuscripts for use with REVTEX 4.0
% Copy this file to another name and then work on that file.
% That way, you always have this original template file to use.
%
% Group addresses by affiliation; use superscriptaddress for long
% author lists, or if there are many overlapping affiliations.
% For Phys. Rev. appearance, change preprint to twocolumn.
% Choose pra, prb, prc, prd, pre, prl, prstab, or rmp for journal
%  Add 'draft' option to mark overfull boxes with black boxes
%  Add 'showpacs' option to make PACS codes appear
%  Add 'showkeys' option to make keywords appear
\documentclass{revtex4}
%\documentclass[aps,prl,preprint,superscriptaddress]{revtex4}
%\documentclass[aps,prl,twocolumn,groupedaddress]{revtex4}
\usepackage[dvipdf]{graphicx}
%\usepackage{dcolumn}

% You should use BibTeX and apsrev.bst for references
% Choosing a journal automatically selects the correct APS
% BibTeX style file (bst file), so only uncomment the line
% below if necessary.
%\bibliographystyle{apsrev}

\setlength{\textwidth}{6.5in}

\begin{document}

% Use the \preprint command to place your local institutional report
% number in the upper righthand corner of the title page in preprint mode.
% Multiple \preprint commands are allowed.
% Use the 'preprintnumbers' class option to override journal defaults
% to display numbers if necessary
%\preprint{}

%Title of paper
\title{HDDM User's Guide}

% repeat the \author .. \affiliation  etc. as needed
% \email, \thanks, \homepage, \altaffiliation all apply to the current
% author. Explanatory text should go in the []'s, actual e-mail
% address or url should go in the {}'s for \email and \homepage.
% Please use the appropriate macro foreach each type of information

% \affiliation command applies to all authors since the last
% \affiliation command. The \affiliation command should follow the
% other information
% \affiliation can be followed by \email, \homepage, \thanks as well.
%\homepage[]{Your web page}
%\thanks{}
%\altaffiliation{}
\author{R.T. Jones}
\affiliation{University of Connecticut}

%Collaboration name if desired (requires use of superscriptaddress
%option in \documentclass). \noaffiliation is required (may also be
%used with the \author command).
%\collaboration can be followed by \email, \homepage, \thanks as well.
\collaboration{GlueX}
\email{richard.t.jones@uconn.edu}
%\noaffiliation

\date{February 9, 2021}

\begin{abstract}
The HDDM (Hierarchical Document Data Model) is an xml schema for expressing
the meaning and relationships of streaming data from scientific instruments.
The design is based on a hierarchical network where each node in the graph
has a single parent node, multiple key-value attributes, and an arbitrary
number of child nodes, similar to a the elements in an xml document.
The model is adapted specifically to the case of repetitive data models
such as appear in the data stream from a high-energy physics experiment.
The representation of the model in xml is an essential feature, although
instantiation in memory does not involve the creation of explicit textual
elements or construction of a Document Object Model (DOM) for the data.
The HDDM toolkit includes tools to express HDDM streams in xml, check their
validity against the schema, and serialize/deserialize from container
objects in memory. Originally written in c, HDDM provides application
programmer interfaces for C++ and python as well. In addition to its own
native data format, applications that use HDDM to access their data can
also read/write standard HDF5 files and ROOT trees.
\end{abstract}

% insert suggested PACS numbers in braces on next line
%\pacs{}
% insert suggested keywords - APS authors don't need to do this
%\keywords{}

\setlength{\topmargin}{0in}

%\maketitle must follow title, authors, abstract, \pacs, and \keywords
\maketitle

% body of paper here - Use proper section commands
% References should be done using the \cite, \ref, and \label commands

%% The normal text is displayed in two-column format, but special
%% sections spanning both columns can be inserted within the page
%% format so that long equations can be displayed. Use
%% sparingly.
%%\begin{widetext}
%% put long equation here
%%\end{widetext}
%
%% figures should be put into the text as floats.
%% Use the graphics or graphicx packages (distributed with LaTeX2e)
%% and the \includegraphics macro defined in those packages.
%% See the LaTeX Graphics Companion by Michel Goosens, Sebastian Rahtz,
%% and Frank Mittelbach for instance.
%%
%% Here is an example of the general form of a figure:
%% Fill in the caption in the braces of the \caption{} command. Put the label
%% that you will use with \ref{} command in the braces of the \label{} command.
%% Use the figure* environment if the figure should span across the
%% entire page. There is no need to do explicit centering.
%
%%\begin{turnpage}
%% Surround figure environment with turnpage environment for landscape
%% figure
%% \begin{turnpage}
%% \begin{figure}
%% \includegraphics{}%
%% \caption{\label{}}
%% \end{figure}
%% \end{turnpage}
%
%% tables should appear as floats within the text
%%
%% Here is an example of the general form of a table:
%% Fill in the caption in the braces of the \caption{} command. Put the label
%% that you will use with \ref{} command in the braces of the \label{} command.
%% Insert the column specifiers (l, r, c, d, etc.) in the empty braces of the
%% \begin{tabular}{} command.
%% The ruledtabular enviroment adds doubled rules to table and sets a
%% reasonable default table settings.
%% Use the table* environment to get a full-width table in two-column
%% Add \usepackage{longtable} and the longtable (or longtable*}
%% environment for nicely formatted long tables. Or use the the [H]
%% placement option to break a long table (with less control than 
%% in longtable).
%
%
%% Surround table environment with turnpage environment for landscape
%% table
%% \begin{turnpage}
%% \begin{table}
%% \caption{\label{}}
%% \begin{ruledtabular}
%% \begin{tabular}{}
%% \end{tabular}
%% \end{ruledtabular}
%% \end{table}
%% \end{turnpage}
%
%% Specify following sections are appendices. Use \appendix* if there
%% only one appendix.
%%\appendix
%%\section{}
%

\tableofcontents
\section{Introduction}

The HDDM toolkit provides the scientist with a means to format streaming data
from a scientific instrument into a structured self-describing byte stream of
binary data that is platform-independent and easy to browse, filter, transform
extend, annotate, and validate using standard xml tools. The purpose of this 
User's Guide is to describe the use and operation of the HDDM tools, describing
how to define a new data model or inspect an existing one, how to create new
HDDM files or read data from existing files. HDDM tools automatically generate
the object classes that represent the data described in the user's model, with
methods to access the object data in memory as well as to serialize/deserialize
themselves between memory and an external byte stream connected to an ordinary
file on disk or to a network socket. The underlying implementation of the i/o
library is in C++, so it provides good performance in terms of data rate to/from
byte streams, with optional on-{}the-{}fly compression/decompression and data 
integrity verification. In addition to ordinary sequential access to data,
random-{}access to records at an arbitrary offset in a stream is also provided
for streams that support random seeks, without the need to read the entire stream.

\section{XML Schemas and HDDM templates}

Every HDDM stream has an associated data model, expressed either in the form of a
standard XML schema, or more compactly, in the form of a HDDM template.  Tools are
provided to translate between the schema and HDDM template description of the model.
A HDDM template is a short xml document that describes the structure of one record
in the HDDM stream. Every HDDM stream has a copy of its template in plain-text
UTF-8 at the beginning, followed by a sequence of data records in a compact binary
format. The template contains all of the information necessary to reconstruct the
original data objects from the serialized records, together with their hierarchical
arrangement. A simple example of a template is given in Fig.~\ref{simple_template}.

\begin{figure}
\begin{minipage}{12cm}
\begin{verbatim}
<?xml version="1.0" encoding="UTF-8"?>
<HDDM class="x" version="1.0" xmlns="http://www.gluex.org/hddm">
  <student name="string" minOccurs="0">
    <enrolled year="int" semester="int" maxOccurs="unbounded">
      <course credits="int" title="string" maxOccurs="unbounded">
        <result grade="string" Pass="boolean" />
      </course>
    </enrolled>
  </student>
</HDDM>
\end{verbatim}
\end{minipage}
\caption{\label{simple_template}
A simple example of a HDDM data model template. A HDDM data stream encodes a
sequence of \texttt{<student>} records, each with the same internal hierarchical
structure defined in the template. The data values designated by simple types,
``int'', ``string'', etc.\ are packed sequentially into the byte stream on output,
and unpacked into memory-resident objects on input.}
\end{figure}

All of the records in the file represent repeats of this basic structure, with 
different values in the data fields. All actual data values are represented as
attributes of tags. Attributes that are assigned type names (``string'', ``int'',
``long'', ``float'', ``double'', ``boolean'', ``anyURI'', and ``Particle\_t'')
are user data. Any other values assigned to attributes other than these simple
types are treated as annotations in the data model, eg.\ to specify the units
assigned to physical values, and do not take up space in the file (other than
in the template header). Some of these literal attributes function as metadata,
eg.\ you might want to add an attribute unit=``GeV'' to document the units used
for other attributes in a tag. Other special attributes like minOccurs/maxOccurs
take special values that tell the data model whether a given element is always
present in every record or may be omitted (minOccurs=``0'') or whether it may
be repeated any number of times (maxOccurs=``unbounded''), as in a standard
xml schema. The top-{}level element is special in that it must always be named
HDDM and have the attributes shown in Fig.~\ref{simple_template}. The class
attribute of the \texttt{<HDDM>} element is any (preferably short) string that
you choose for the family of data models you are creating. Choose a short, unique
name for your class because it is used in the type names of user objects that are
defined written by the HDDM user library. Its purpose is to prevents collisions
between different HDDM stream types that may coexist in a single application.

Templates provide an intuitive way of specifying the structure of a data record
in a HDDM stream. For most users, this is all they need to know about in order
to define their data models. For those familiar with XML schema validation,
there is a more formal way to specify the structure of an xml document which is
called a {\em xml schema}. HDDM uses schemas in two different ways. The first
is to specify the structure of the templates themselves. The template shown in
Fig.~\ref{simple_template} conforms to a schema called \url{http://www.gluex.org/hddm}.
This is not a URL to anywhere; it is a URI known as an {\em xml namespace}, as
suggested by the name of the \texttt{xmlns} attribute in the HDDM tag of the template.
The schema for this document type is found in hddm\_schema.xsl in the schema
directory of the distribution. The second use of schemas is related to the fact
that every record in a HDDM stream is a valid xml fragment corresponding to a
schema against which it can be validated. The HDDM toolkit provides a pair of
tools {\em hddm-{}schema} and {\em schema-{}hddm} that convert back and forth
between templates and schema. The two are equivalent ways of representing the
same information about the structure of a HDDM record, with the schema being
more complete and standards-{}based, while the template is shorter and more 
intuitive to most users. Schemas provide a much more general set of constraints
that can be expressed for the data and relationships between them, but 
experience has shown that their practical use for this purpose is limited to
special instances where standards-based data validation must be performed.
The remainder of this document deals mainly only with templates.

\section{How to get started}

The HDDM toolkit is distributed as a github repository \url{https://github.com/rjones30/HDDM}.
Instructions for how to download and build HDDM are given in the INSTALL file
provided at the top level of the download tree. The HDDM tools are installed
by the installation procedure into the bin directory under the installation
base. Before continuing to read this document, make sure that the basic HDDM tools
including hddm-{}xml, xml-{}hddm, hddm-{}c, hddm-{}cpp, hddm-{}py, and xml-{}xml
are in your shell PATH. These tools are not the HDDM libraries themselves, but
the code generators you need to construct user-callable libraries from your 
HDDM template.

If you already have a HDDM data file that you want to read, you can generate
the i/o user library that you can use to read from it and optionally to write
a selection of the records to a new HDDM output file. The template that the
code generators need to generate the user library is present in the header of
the HDDM file that you want to read. Simply providing the data file as input
to \texttt{hddm-c} will generate c header and implementation files that you
need to include on the compiler command line together with your c application
code for your project, and similarly, \texttt{hddm-cpp} in the case of C++
applications, or \texttt{hddm-py} to generate a python module. Of the three
supported programming languages, the python implementation is the least
verbose and most readable, so it is recommended as a starting point for
someone experimenting with HDDM.

Independent of any user programs or language-{}specific API, the HDDM toolkit
provides two tools that can be used to read and write HDDM files directly from
the command line. The following command accepts any valid HDDM file as input
and prints the contents of the file in plain-{}text xml to standard output.

\vspace{0.5cm}
\begin{minipage}{12cm}
\begin{verbatim}
$ hddm-xml [-n <count>] [-o <output.xml>] <datafile.hddm> [...]
\end{verbatim}
\end{minipage}
\vspace{0.5cm}

The reverse action is provided by the \texttt{xml-{}hddm} tool.

\vspace{0.5cm}
\begin{minipage}{12cm}
\begin{verbatim}
$ xml-hddm [-n <count>] -t template.xml <input.xml> [...]
\end{verbatim}
\end{minipage}
\vspace{0.5cm}

The full XML rendition of a data file with many records is highly verbose,
which makes the plain-text xml rendering of a HDDM stream of limited practical
interest, except as a means to visually browse the data, or to make small changes
using a text editor. The reversibility of the conversion between xml and HDDM
representations can be useful in cases where one might doubt the fidelity of the
encoding being used by HDDM. These two tools do not require any 
compile-{}and-{}link step each time the template is changed, so they are very
useful to quickly inspect the contents of a HDDM file. Keep them handy when
working through the language-{}specific procedures below.

\section{HDDM in python}

If you have access to a HDDM file that was already written, copy it into your
work directory and use it as a template for building a python module to access the
model data as python list objects. Otherwise, the HDDM package distribution directory
contains a simple example in \texttt{models/exam1x.hddm} that you can use for this
purpose. Copy the HDDM file you are using for this test into a new project directory,
and enter the following commands to build the python module for this data model. If
you encounter the error, ``command not found'', make sure that the bin directory
where you installed the HDDM package is somewhere in your shell PATH.

\vspace{0.5cm}
\begin{minipage}{12cm}
\begin{verbatim}
$ hddm-cpp exam1x.hddm  # builds the underlying C++ library
$ hddm-py exam1x.hddm   # builds the python interface
$ python setup_hddm_x.py # creates the module hddm_x
\end{verbatim}
\end{minipage}
\vspace{0.5cm}

In this example, I assigned `x' as the HDDM {\em class} abbrivation (see the HDDM
tag in the template header). You should change it to whatever class abbreviation
you choose for your HDDM data model. The above steps should create a python module
in the form of a shared library that starts with \texttt{hddm\_x} in your project
directory. Copy that module to a directory in your PYTHONPATH where you usually
place your private python modules, or add your project directory to your 
PYTHONPATH.

Execute the following interactive python script to print the contents of the
example HDDM file in plain text.

\vspace{0.5cm}
\begin{minipage}{12cm}
\begin{verbatim}
import hddm_x
for rec in hddm_x.istream("exam1x.hddm"):
   print(rec)
\end{verbatim}
\end{minipage}
\vspace{0.5cm}

To see the same data printed out as a properly formatted xml document, replace
the \texttt{print(rec)} in the above python HDDM reader with 
\texttt{print(rec.toXML())}. 
If the above command generated no output then your input HDDM file is empty, as it
would be if you used the example input file \texttt{models/exam1x.hddm}. After you
have written some data to an HDDM file, as exlained in the next section, come back
and try it again. The full set of methods and attributes supported by the python
module is displayed by the command, ``\texttt{pydoc hddm\_x}''.

\subsection{writing HDDM files in python}

For this example, let us continue using the same template as was used in the
example python HDDM reader above. You should already have built and installed
the hddm\_x python module and installed it in your PYTHONPATH, using the build
steps listed above. Execute the following python script to write a new output
HDDM file from scratch, using and some test user data.

\vspace{0.5cm}
\begin{minipage}{12cm}
\begin{verbatim}
import hddm_x
ofs = hddm_x.ostream('exam2x.hddm')
xrec = hddm_x.HDDM()
student = xrec.addStudents()
student[0].name = 'Humphrey Gaston'
enrolled = student[0].addEnrolleds()
enrolled[0].year = 2005
enrolled[0].semester = 2
course = enrolled[0].addCourses(3)
course[0].credits = 3
course[0].title = 'Beginning Russian'
result = course[0].addResults()
result[0].grade = 'A-'
result[0].Pass = True
course[1].credits = 1
course[1].title = 'Bohemian Poetry'
result = course[1].addResults()
result[0].grade = 'C'
result[0].Pass = 1
course[2].credits = 4
course[2].title = 'Developmental Psychology'
result = course[2].addResults()
result[0].grade = 'B+'
result[0].Pass = True
ofs.write(xrec)
\end{verbatim}
\end{minipage}
\vspace{0.5cm}

This script generates a new HDDM file called \texttt{exam2x.hddm}. Now
running the 3-{}line python reader from the previous section on 
\texttt{exam2x.hddm} should yield the following output.

\vspace{0.5cm}
\begin{minipage}{12cm}
\begin{verbatim}
HDDM
  student name="Humphrey Gaston"
    enrolled semester=2 year=2005
      course credits=3 title="Beginning Russian"
        result Pass=false grade="A-"
      course credits=1 title="Bohemian Poetry"
        result Pass=false grade="C"
      course credits=4 title="Developmental Psychology"
        result Pass=false grade="B+"
\end{verbatim}
\end{minipage}
\vspace{0.5cm}

The example writer above began by creating a new record by calling
the \texttt{HDDM()} record constructor. Then it populated the structure
top-{}down, calling \texttt{add}XXXs() methods for each tag XXX under
that, where XXXs refers to the name of the tag element in the template
transformed into its capitalized-{}plural form. The \text{add}XXXs()
methods take a single optional \texttt{int} argument, which
is the number of copies of that element that need to be added (default 1).
They return a list that can be indexed in the usual python fashion to give
access to the individual members of the list. Each of these has \texttt{add}XXXs()
methods for each of its contents, and so on down the tree. You can omit
whole branches of the tree by simply not calling the corresponding 
\texttt{add}XXXs() method. Xml rules require that you specify
minOccurs=``0'' in the template for the containing tag if you plan to make that
subtree optional. As soon as a new element list is created, you can fill in the
values of its attributes using simple assignment semantics, as illustrated
in the example. The names of the python data members are the same as the
names of the attributes in the template.

\subsection{reading HDDM files in python}

For this illustration, you are assumed to have created the file exam2x.hddm using
the writer described in the previous section. The following python program
lets you open this file and extract bits of information from the first record,
writing a summary report at the end. Of course, in actual practice, a HDDM
file would contain many records and the analysis would loop over many instances
of student.

\vspace{0.5cm}
\begin{minipage}{12cm}
\begin{verbatim}
import hddm_x
ifs = hddm_x.istream("exam2x.hddm")
xrec = ifs.read()
total_enrolled = 0
total_courses = 0
total_credits = 0
total_passed = 0
for course in xrec.getCourses():
   total_courses += 1
   if course.getResult().Pass:
      if course.year > 1992:
         total_credits += course.credits
      total_passed += 1
   total_enrolled += 1
   studentname = course.name

print(studentname, "enrolled in", total_courses, " courses",
       "and passed" , total_passed, "of them,\n",
       "earning a total of", total_credits, "credits.")
\end{verbatim}
\end{minipage}
\vspace{0.5cm}

Running the above code should produce output like the following:

\vspace{0.5cm}
\begin{minipage}{12cm}
\begin{verbatim}
Humphrey Gaston enrolled in 3 courses and passed 3 of them,
 earning a total of 8 credits.
\end{verbatim}
\end{minipage}
\vspace{0.5cm}

In addition to each tag supporting the lookup (via getXXXs methods) of the
tags immediately appearing under it in the template hierarchy, the 
top-{}level HDDM record provides global getXXXs methods for every tag
throughout the hierarchy, and returns all instances of a given tag that
appear anywhere in the record, in the order of their appearance. The
istream object itself also functions as an iterable in python so the
construct, \texttt{for rec in hddm\_x.istream('exam2x.hddm'):} would
look over all records in the input file, assigning the \texttt{rec}
iteration variable to each record as it is read from the input stream.
Likewise, each call to method getXXXs() returns a python list of tag
element objects that is iterable using the usual python \texttt{for}
semantics, as illustrated for xrec.getCourses() above. As before, the
individual attributes of each tag instance are accessed as plain data 
members of their host object. The standard python list functions (eg.
len(list), str(list), repr(list)) all work as expected for these hddm
tag list objects returned by getXXXs() method. These natural python
iteration and accessor semantics provide a quick-{}and-{}simple
prototyping framework for analysis of repetitive experimental data.

A slightly more complex example of reading and writing HDDM streams
based on this example template is found in the distribution under
\texttt{examples/exam2.py}.

\subsection{advanced features of the python API}

See section \ref{Advanced_features} {Advanced features} below.

\section{HDDM in C++}

If you have access to a HDDM file that was already written, copy it into your
work directory and use it as a template for building a C++ library to access
the model data as C++ objects. Otherwise, the HDDM package distribution
directory contains a simple example in \texttt{models/exam1x.hddm} that you
can use for this purpose. The following commands build the C++ library
corresponding to your HDDM model.

\vspace{0.5cm}
\begin{minipage}{12cm}
\begin{verbatim}
$ hddm-cpp exam1x.hddm
$ g++ -c  hddm_x++.cpp -I $HDDM_INSTALL_DIR/include \
  $HDDM_SOURCE_DIR/XString.cpp $HDDM_SOURCE_DIR/XParsers.cpp \
  $HDDM_SOURCE_DIR/md5.c -I$XERCESCROOT/include \
  -L $XERCESCROOT/lib -l xerces-c \
  -L $HDDM_INSTALL_DIR/lib -lxstream -lz -lbz2
\end{verbatim}
\end{minipage}
\vspace{0.5cm}

If the environment variables in this command are not defined in your shell
environment, define them or replace them with the appropriate values.

\subsection{writing HDDM files in C++}

This section turns once again to the example template \texttt{exam1x.hddm} used
earlier under the python hddm writer section. Having already built a C++ library
against this template, now it is time to write a user application that uses the
library to create HDDM output according to the template. Open a new C++ source
file in an editor and cut/paste the contents of the box below into it, then save it.

\vspace{0.5cm}
\begin{minipage}{12cm}
\begin{verbatim}
#include <fstream>
#include "hddm_x.hpp"
int main()
{
   // build the nodal structure for this record and fill in its values
   hddm_x::HDDM xrec;
   hddm_x::StudentList student = xrec.addStudents();
   student().setName("Humphrey Gaston");
   hddm_x::EnrolledList enrolled = student().addEnrolleds();
   enrolled().setYear(2005);
   enrolled().setSemester(2);
   hddm_x::CourseList course = enrolled().addCourses(3);
   course(0).setCredits(3);
   course(0).setTitle("Beginning Russian");
   course(0).addResults();
   course(0).getResult().setGrade("A-");
   course(0).getResult().setPass(true);
   course(1).setCredits(1);
   course(1).setTitle("Bohemian Poetry");
   course(1).addResults();
   course(1).getResult().setGrade("C");
   course(1).getResult().setPass(1);
   course(2).setCredits(4);
   course(2).setTitle("Developmental Psychology");
   course(2).addResults();
   course(2).getResult().setGrade("B+");
   course(2).getResult().setPass(true);

   std::ofstream ofs("exam2x.hddm");
   hddm_x::ostream ostr(ofs);
   ostr << xrec;
   xrec.clear();
   return 0;
}
\end{verbatim}
\end{minipage}
\vspace{0.5cm}

Save this C++ program to a file named \texttt{write\_exam.cpp} and compile it 
into an executable using a command like the following.

\vspace{0.5cm}
\begin{minipage}{12cm}
\begin{verbatim}
$ g++ -o write_exam write_exam.cpp hddm_x++.o -I. -I $HDDM_INSTALL_DIR/include \
  $HDDM_SOURCE_DIR/XString.cpp $HDDM_SOURCE_DIR/XParsers.cpp \
  $HDDM_SOURCE_DIR/md5.c -I$XERCESCROOT/include \
  -L $XERCESCROOT/lib -l xerces-c \
  -L $HDDM_INSTALL_DIR/lib -lxstream -lz -lbz2
\end{verbatim}
\end{minipage}
\vspace{0.5cm}

The paths listed in the compilation command line may need to be customized
for your own build environment. Once it completes successfully, will find
the executable \texttt{write\_exam} in the working directory. Run it as
\texttt{./write\_exam2} and it should create a new HDDM file called 
\texttt{exam2x.hddm}. Running \texttt{hddm-{}xml write\_exam2x.hddm}
should produce output like the following.

\vspace{0.5cm}
\begin{minipage}{12cm}
\begin{verbatim}
HDDM
  student name="Humphrey Gaston"
    enrolled semester=2 year=2005
      course credits=3 title="Beginning Russian"
        result Pass=false grade="A-"
      course credits=1 title="Bohemian Poetry"
        result Pass=false grade="C"
      course credits=4 title="Developmental Psychology"
        result Pass=false grade="B+"
\end{verbatim}
\end{minipage}
\vspace{0.5cm}

The structure of the output record you are writing is already known to
the program because it has been built against the `x' template. All that
the writer needs to do is to add the data elements and assign the values of
the attributes. You begin by creating an empty record by calling the 
\texttt{HDDM()} default constructor. Then you populate the structure
top-{}down by calling \texttt{addXXXs()} methods for each tag \texttt{XXX}
under that. The name \texttt{XXX}s is the name of the tag element in the
template in a capitalized-{}plural form. The \texttt{addXXXs()} methods
take a single optional int argument, which is the number of instances of
that element to be added to the containing element (default is 1). They
return a subclass of \texttt{std::list} that can be iterated over in the
usual fashion, or indexed with \texttt{operator()(int)} to access the
individual members of the list. Each of these has \texttt{addXXXs()}
methods for each of its contents, and so on down the tree. You can omit
whole branches of the tree by simply not calling the corresponding
\texttt{addXXXs()} method, although xml rules require that you specify
minOccurs=``0'' for the containing tag in the template if you plan to do
that. As soon as a new element list is constructed, you can fill in the
values of its object attributes using set{\mbox{$<$}}attname{\mbox{$>$}}
methods, as illustrated in the example, where {\mbox{$<$}}attname{\mbox{$>$}}
is a capitalized version of the names of the attribute in the template.

\subsection{reading HDDM files in C++}

This section assumes that you have created the file exam2x.hddm using the
procedure described in the previous section. The following C++ program 
opens this file and extracts bits of information from the first record,
and writes a summary report. Of course, in actual practice such a data
file would probably contain many records, and the analysis would loop
over many instances of student.

\vspace{0.5cm}
\begin{minipage}{12cm}
\begin{verbatim}
#include <fstream>
#include "hddm_x.hpp"
int main()
{
   std::ifstream ifs("exam2x.hddm");
   hddm_x::HDDM xrec;
   hddm_x::istream istr(ifs);
   istr >> xrec;
   hddm_x::CourseList course = xrec.getCourses();
   int total_courses =course.size();
   int total_enrolled = 0;
   int total_credits = 0;
   int total_passed = 0;
   hddm_x::CourseList::iterator iter;
   for (iter = course.begin(); iter != course.end(); ++iter) {
      if (iter->getResult().getPass()) {
         if (iter->getYear() > 1992) {
            total_credits += iter->getCredits();
         }
         ++total_passed;
      }
   }
   std::cout << course().getName() << " enrolled in "
             << total_courses << " courses "
             << "and passed " << total_passed << " of them, " << std::endl
             << "earning a total of " << total_credits
             << " credits." << std::endl;
   return 0;
}
\end{verbatim}
\end{minipage}
\vspace{0.5cm}

Running the above code should produce output like the following:

\vspace{0.5cm}
\begin{minipage}{12cm}
\begin{verbatim}
Humphrey Gaston enrolled in 3 courses and passed 3 of them,
earning a total of 8 credits.
\end{verbatim}
\end{minipage}
\vspace{0.5cm}

See section \ref{Advanced_features} {Advanced features} below.

\section{HDDM in c}

If you have access to a HDDM file that was already written, copy it into your
work directory and use it as a template for building a python module to access
the model data as c struct records. Otherwise, the HDDM package distribution directory
contains a simple example in \texttt{models/exam1x.hddm} that you can use for
this purpose. The following commands build the c library that you will need
to read and write HDDM streams that conform to this template.

\vspace{0.5cm}
\begin{minipage}{12cm}
\begin{verbatim}
$ hddm-c exam1x.xml
$ gcc -c hddm_x.c $HDDM_SOURCE_DIR/md5.c \
  -I$XERCESCROOT/include -L $XERCESCROOT/lib -l xerces-c \
  -L$HDDM_INSTALL_DIR/lib -lxstream -lz -lbz2
\end{verbatim}
\end{minipage}
\vspace{0.5cm}

\subsection{writing HDDM files in c}

This example turns once again to the template \texttt{exam1x.hddm} that is included
with the source distribution. Use the instructions in the previous section to build
the \texttt{hddm\_x} c API library, then create a new main program source file and
cut/paste the code below into it, then save it.

\vspace{0.5cm}
\begin{minipage}{12cm}
\begin{verbatim}
#include "hddm_x.h"

int main()
{
   x_iostream_t* fp;
   x_HDDM_t* exam2;
   x_Student_t*  student;
   x_Enrolleds_t* enrolleds;
   x_Courses_t* courses;
   x_Result_t* result;
   string_t name;
   string_t grade;
   string_t course;

   // first build the complete nodal structure for this record
   exam2 = make_x_HDDM();
   exam2->student = student = make_x_Student();
   student->enrolleds = enrolleds = make_x_Enrolleds(99);
   enrolleds->mult = 1;
   enrolleds->in[0].courses = courses = make_x_Courses(99);
   courses->mult = 3;
   courses->in[0].result = make_x_Result();
   courses->in[1].result = make_x_Result();
   courses->in[2].result = make_x_Result();

   // now fill in the attribute data for this record
   name = malloc(30);
   strcpy(name,"Humphrey Gaston");
   student->name = name;
   enrolleds->in[0].year = 2005;
   enrolleds->in[0].semester = 2;
   courses->in[0].credits = 3;
   course = malloc(30);
   courses->in[0].title = strcpy(course,"Beginning Russian");
   grade = malloc(5);
   courses->in[0].result->grade = strcpy(grade,"A-");
   courses->in[0].result->Pass = 1;
   courses->in[1].credits = 1;
   course = malloc(30);
   courses->in[1].title = strcpy(course,"Bohemian Poetry");
   grade = malloc(5);
   courses->in[1].result->grade = strcpy(grade,"C");
   courses->in[1].result->Pass = 1;
   courses->in[2].credits = 4;
   course = malloc(30);
   courses->in[2].title = strcpy(course,"Developmental Psychology");
   grade = malloc(5);
   courses->in[2].result->grade = strcpy(grade,"B+");
   courses->in[2].result->Pass = 1;

   // now open a file and write this one record into it
   fp = init_x_HDDM("exam2.hddm");
   flush_x_HDDM(exam2,fp);
   close_x_HDDM(fp);

   return 0;
}
\end{verbatim}
\end{minipage}
\vspace{0.5cm}

Save this c program to a file called \texttt{write\_exam2.c} and compile it
into an executable using a command like the following.

\vspace{0.5cm}
\begin{minipage}{12cm}
\begin{verbatim}
$ gcc -o write_exam2 write_exam2.c hddm_x.o -I. -I $HDDM_INSTALL_DIR/include \
  -I$XERCESCROOT/include -L $XERCESCROOT/lib -l xerces-c \
  -L $HDDM_INSTALL_DIR/lib -l xstream -lbz2 -lz
\end{verbatim}
\end{minipage}
\vspace{0.5cm}

The shell environment variables containing the package installation paths in
the above compile command may need to be customized for your own environment.
Once it completes successfully, you will find the executable \texttt{write\_exam2}
in the working directory. Run it as \texttt{./write\_exam2} and it should create
a new HDDM file called \texttt{exam2.hddm}. Running
\texttt{hddm-{}xml write\_exam2.hddm} should produce output like the following.

\vspace{0.5cm}
\begin{minipage}{12cm}
\begin{verbatim}
HDDM
  student name="Humphrey Gaston"
    enrolled semester=2 year=2005
      course credits=3 title="Beginning Russian"
        result Pass=false grade="A-"
      course credits=1 title="Bohemian Poetry"
        result Pass=false grade="C"
      course credits=4 title="Developmental Psychology"
        result Pass=false grade="B+"
\end{verbatim}
\end{minipage}
\vspace{0.5cm}

This example explains most of what you need to know to set up HDDM c-structs
in memory, and write them to an output file. All storage for HDDM data is
allocated on the heap. Most of this allocation is carried out automatically
by the \texttt{make\_x\_XXXs()} functions, although for strings (char arrays)
the user needs to allocate initial storage for the values. Memory pointed to
by the pointers returned by the \texttt{make\_x\_XXXs()} functions is owned
by the user code until the pointer to it gets assigned to a HDDM struct member
that is designated in the data model to hold it. After that, the memory is
owned by the top-{}level HDDM containing record object, and should only be
freed by calling the \texttt{flush\_x\_HDDM()} method. Calling
\texttt{flush\_x\_HDDM(record, fp)} with its second argument (FILE*) open
to an output file causes the record to be written to the output file.
Calling it as \texttt{flush\_x\_HDDM(record,0)} causes it to bypass the
output serialization step. Either way, \texttt{flush\_x\_HDDM()} frees all
memory owned by the HDDM record, discarding its contents, before it returns.

The structure of the output record you are writing is already known to the
program because the struct is specified according to the template. All that
you need to do is to fill in the elements and assign the values of the attributes.
You begin by creating an empty record by calling \texttt{make\_x\_HDDM()}.
Then you populate the structure top-{}down by calling \texttt{make\_x\_XXXs()}
for each tag \texttt{XXX} and assigning pointers to each one into the 
appropriate structure element of the parent element. The name \texttt{XXXs}
is the name of the tag element in the template in a capitalized-{}plural form.
The \texttt{addXXXs()} methods take a single optional int argument, which is
the number of copies of that element that need to be added (default is 1). 
They return a pointer to an array of struct pointers which can be indexed in
the usual c-{}fashion to access the individual members of the array. Each of
the contained elements within a given host tag have a corresponding pointer
in the host struct that must be assigned in the user code to the value returned
by the \texttt{make\_x\_XXXs()} function, as illustrated. Any such pointers
that are not assigned remain null (initialized by \texttt{make\_x\_XXXs}) and
represent parts of the template tree that are missing from the record. This
is a perfectly valid HDDM record, but user code must check for the NULL
pointer condition before trying to dereference it since c has no automatic
checking of the validity of pointers. As soon as a new struct array element is
created, you can fill in the values of its attribute members using direct
assignment semantics, as illustrated in the example above. Any values that
are not explicitly assigned remain at the default values, typically zero or null.

\subsection{reading HDDM files in c}

This section assumes that you have created the file exam2.hddm using the 
instructions in the previous section. The following c program opens this file
and extracts bits of information from the first record, writing a summary
report at the end. Of course, in actual practice a HDDM file would probably
contain many records, and the analysis would loop over many instances student.

\vspace{0.5cm}
\begin{minipage}{12cm}
\begin{verbatim}
#include "hddm_x.h"

int main()
{
   x_iostream_t* fp;
   x_HDDM_t* exam2;
   x_Student_t* student;
   x_Enrolleds_t* enrolleds;
   int enrolled;
   x_Courses_t* courses;
   int course;
   int total_enrolled,total_courses,total_credits,total_passed;

 // read a record from the file
   fp = open_x_HDDM("exam2.hddm");
   if (fp == NULL) {
      printf("Error - could not open input file exam2.hddm\n");
      exit(1);
   }
   exam2 = read_x_HDDM(fp);
   if (exam2 == NULL) {
      printf("End of file encountered in hddm file exam2.hddm, quitting!\n");
      exit(2);
   }

   // examine the data in this record and print a summary
   total_enrolled = 0;
   total_courses = 0;
   total_credits = 0;
   total_passed = 0;
   student = exam2->student;
   enrolleds = student->enrolleds;
   total_enrolled = enrolleds->mult;
   for (enrolled=0; enrolled<total_enrolled; ++enrolled) {
      courses = enrolleds->in[enrolled].courses;
      total_courses += courses->mult;
      for (course=0; course<courses->mult; course++) {
         if (courses->in[course].result->Pass) {
            if (enrolleds->in[enrolled].year > 1992) {
               total_credits += courses->in[course].credits;
            }
            ++total_passed;
         }
      }
   }
   printf("%s enrolled in %d courses.\n",student->name,total_courses);
   printf("He passed %d of them, earning a total of %d credits.\n",total_passed,total_credits);

   flush_x_HDDM(exam2,0);  // don't do this until you are done with exam2
   close_x_HDDM(fp);
   return 0;
}
\end{verbatim}
\end{minipage}
\vspace{0.5cm}

Running the above code should produce output like the following:

\vspace{0.5cm}
\begin{minipage}{12cm}
\begin{verbatim}
Humphrey Gaston enrolled in 3 courses and passed 3 of them,
earning a total of 8 credits.
\end{verbatim}
\end{minipage}
\vspace{0.5cm}

Having read the section above on how to write HDDM records using the c
interface, it should be easy to understand the meaning of the above code.
The \texttt{read\_x\_HDDM()} call allocates all of the memory needed to
stand up the full record hierarchy in memory. The \texttt{flush\_x\_HDDM(record,0)}
call at the end of the loop ensures that all of this memory gets recycled
to the heap before the next record is read in. Accessing leaf elements that
are deep inside the HDDM template hierarchy can only be achieved by traversing
all of the nodes in the tree above, which makes a simple data mining operation
somewhat verbose, as illustrated in the above example, although it still scales
well because the model is hierarchical, not a linked list. If you are unsure
about how to do something, browsing within the header file is probably not
going to be very rewarding because all of the internal functionality of the
logic that supports the serialization/deserialization of the data is exposed
there. However, the user API is very simple. Access to the data-bearing attributes
is through direct struct member access. Only the \texttt{make\_x\_XXXs} functions
and the input/output functions (open, close, read, flush, skip) should be called
by the user; all the rest are for internal use. As is the case for for all of the
other API's, the template itself should be the only documentation
you need to consult when writing code that interacts with HDDM data.

\subsection{advanced features of the c API}

The c API is no longer in active development. It is supported only for legacy
applications that rely on it. The features described in the \ref{Advanced_features}
{Advanced Features} section below are not available using the c API. The only things
that are ensured with regard to ongoing support of the c API is that it can read
the streams that it writes based on any valid HDDM template, and that HDDM files
written using the c-API can be read by applications built using any of the other
API's. The converse of the last statement is not guaranteed to
be true in all cases. If an input file is not readable by the c-{}API, it prints
a polite error message reporting this fact and quietly exits.

\section{Advanced features}\label{Advanced_features}

\subsection{on-{}the-{}fly compression/decompression}

HDDM streams added support for on-{}the-{}fly compression on output (and 
decompression on input) with the introduction of the C++ API. Because the 
python API is a thin wrapper around the C++ classes, it also supports this
feature. Compression can obviously only be controlled when the stream is
being written. It can be switched on and off at any time after the stream
is opened, either before the first record is written or any time thereafter.
Whenever it is turned on or off, a small marker is written to the byte
stream that tells the reader when to enable/disable decompression on the
input stream. These transitions occur silently during input; no user
action is needed, and no log messages are automatically generated. Two
compression options are supported.

\begin{enumerate}
\item{\bf bzip2 compression} -{}
This option offers the best compression ratio, a factor of about 2.5 for
particle physics experimental Monte Carlo data. It is also the most 
expensive in terms of cpu time needed during output. Cpu demand for
decompression is much lower, more than an order of magnitude.
\item{\bf zlib compression} -{}
This option offers somewhat lower compression ratios, a factor of about 
1.9 for particle physics experimental Monte Carlo data. However, it is
also much less expensive in terms of cpu time than bzip2, by more than
a factor 3. Cpu demand for decompression is much lower than compression,
as is usually the case with codecs.
\end{enumerate}

Both options are provided because each has its strengths and weaknesses
in terms of cost/performance, and their relative behaviors may be quite
different for different data models. Another factor to take into 
consideration when deciding which compression algorithm to use, if at all,
is the implications of the compression block size on the efficiency for
random access to records in the stream. For more information about random
access, see the relevant section below. If the stream is uncompressed,
random access to a particular record generates a read starting at the
beginning of that specific record and only taking in the contents of 
that record, whereas if the stream is compressed, the entire compression
block containing the record must be decompressed before the data for the
desired record can be pulled in. The compression blocks for bzip2 
compression are almost 1MB in size, whereas the zlib blocks are much
smaller, around 32KB. There is no general answer to the question of which
compression option is best. The person producing the data should consider
what the most likely scenarios are for reading the data, and weigh the
costs and benefits of compression before making this decision.

In the C++ API, the HDDM namespaces have defined the following constants
to distinguish different states of compression:

\begin{itemize}
\item \texttt{k\_no\_compression}
\item \texttt{k\_z\_compression}
\item \texttt{k\_bz2\_compression}
\end{itemize}

One of these three constants should be passed as mode to the 
\texttt{setCompression(mode)} method of the \texttt{hddm\_x::ostream}
class to initialize or change the compression state of any given output
stream. All records written after this method is called will reflect the
change. The present compression mode of either an input or output hddm
stream can be queried by calling method \texttt{getCompression()}.
The return value (int) can be compared with the three constants above
to determine which of the three modes is presently enabled.

In python HDDM modules, stream objects of class \texttt{hddm\_x.istream}
and \texttt{hddm\_x.ostream} support selection and sensing of the current
compression mode by exposing read/write attribute \texttt{compression}.
The named constants listed above are defined within the \texttt{hddm\_x}
module namespace. Setting bz2 compression on an open ostream would look
like, \texttt{fout.compression = hddm\_x.k\_bz2\_compression}.

\subsection{on-{}the-{}fly data integrity checks}

HDDM streams added support for on-{}the-{}fly data integrity checks with
the introduction of the C++ API. Because the python API is a thin wrapper
around the C++ classes, it also supports this feature. Data integrity
verification works by the writer computing a hash value on each output
record and storing it as part of the output stream, which the reader then
pulls off the stream and uses to verify the integrity of the data is reads
from the stream. Two 32-{}bit hash algorithms are currently supported by HDDM.

\begin{enumerate}
\item{\bf CRC32} -{} the 32-{}bit cyclic redundancy check algorithm
\item{\bf MD5} -{} the MD5 one-{}way hash algorithm
\end{enumerate}

CRC is considered in cryptographic circles as an error detection algorithm,
meaning that a single bit change in the data record will result in a
change in the 32-{}bit code, and it is very rare that a combination of
errors cancels out and generates the same crc as the original data. This
is probably all we need for our data, and it is much faster to compute than
MD5. MD5 is called a one-{}way hash in cryptographic jargon, which means that
a single bit change in the data record will be reflected in a {\em vastly}
different value for this 32-{}bit code, with approximately 50\% of the bits
changing in the hash as a result of a single bit-{}flip in the input. One might
consider this marginally better for error detection in a random byte stream,
but it is more expensive to compute than a CRC code. Neither MD5 nor CRC32
options result in any noticeable overhead in the context of any models tested
so far.

In the C++ API, the HDDM namespaces have defined the following constants to
distinguish different states of data integrity checking:

\begin{itemize}
\item \texttt{k\_no\_integrity}
\item \texttt{k\_crc32\_integrity}
\item \texttt{k\_md5\_integrity}
\end{itemize}

One of these three values should be passed as mode to the 
\texttt{setIntegrityChecks(mode)} method of the \texttt{hddm\_x::ostream}
class to change the current state of the output stream. All records written
after this method is called will reflect this change. The present integrity
checking mode of either an input or output HDDM stream can be queried by
calling method \texttt{getIntegrityChecks()}. The return value (int) can
be compared with the three constants above to determine which of the three
integrity checking modes is presently enabled.

In python HDDM modules, stream objects of class \texttt{hddm\_x.istream}
and \texttt{hddm\_x.ostream} support the same interface by exposing read/write
attribute \texttt{integrity}. The named constants listed above are defined
within the \texttt{hddm\_x} module namespace. Setting CRC32 compression on
an open ostream might look like, \texttt{fout.integrity = hddm\_x.k\_crc32\_integrity}.

\subsection{random access to HDDM records}

HDDM streams added support for random access on input with the introduction
of the python API. Because the python API is a thin wrapper around the C++
classes, it is also supported by the C++ API. Random-{}access writing to HDDM
streams is not supported; the access point for output streams is always
positioned after the end of the previous output record. Random-access reads
are supported on any input stream that supports repositioning. To succeed,
the random access position must have been generated by a previous call to
the \texttt{getPosition()} method of the same HDDM stream, either during the
initial phase when the stream was being written, or during a subsequent read
pass over the same stream. The \texttt{getPosition()} query returns an opaque
value representing a point in the stream either at the beginning of the first
record, or immediately after the last valid record read or written on the 
stream. Random access to individual records in the input HDDM stream can
take place in any order, and involve displacements either forward or reverse
from the position of last access. 

Attempts to access a stream at an uninitialized position, or at a position
that was generated on a different HDDM stream, will result in unpredictable
behavior, most likely a segmentation fault upon the next attempt to read
from the stream. The following three integer values are needed to define
a stream position.

\begin{enumerate}
\item {\bfseries block\_start} (uint64\_t) -{} absolute stream position 
(std::streampos value) of the beginning of the block containing the record
\item {\bfseries block\_offset} (uint32\_t) -{} offset with the block to
the start of the designated record, or 0 if compression is disabled
\item {\bfseries block\_status} (uint32\_t) -{} complete state (compression,
integrity, other information about the stream state at this position)
\end{enumerate}

If a database were used to store a map of valid positions for a set of HDDM
files, a minimum field width of 128 bits would be needed. Of course, you might
want to save the name and creation date of the input file that the positions
apply to, so that you do not accidentally try to apply them to a different
file than they were created for. If the stream is uncompressed then 
\texttt{block\_offset}=0, but still \texttt{block\_start} and 
\texttt{block\_status} would be needed. The \texttt{block\_status} value is
typically the same for all positions in a given file or dataset, so in most
cases only a single value for that variable needs to be kept, together with
a list of the starts and offsets for the given file.


The object class \texttt{hddm\_x::streamposition} is used to hold stream
position information. Public data members with the names listed above are
exposed for members of the streamposition class. Both \texttt{hddm\_x::istream}
and \texttt{hddm\_x::ostream} classes have \texttt{getPosition()} members that
return a streamposition value. It can either be recorded by saving the values
of its three data members, or by keeping the object in memory and passing it
to the corresponding \texttt{istream::setPosition(streamposition)} method
called on an istream that is (presumably) open for input on the same file.
If the 3 values are stored, they can later be quickly turned back into a
streamposition object using the constructor 
\texttt{streamposition(start,offset,status)}.

HDDM files that were written since this feature was introduced are marked
with the capability to support random access. To check if a given file that
has been opened for input on a \texttt{hddm\_x::istream} supports random access,
simply call method \texttt{getPosition()} within a try-{}catch block and catch
the RuntimeError that is thrown if the input does not support this feature.

Support in the python API for random access follows closely the scheme
described above for C++. The \texttt{hddm\_x.istream} and 
\texttt{hddm\_x.ostream} classes both have read/write data members called
\texttt{position} that reference objects of type \texttt{hddm\_x.streamposition}.
These objects can be saved and then later assigned to an
\texttt{hddm\_x.istream} opened on the same file to seek to the same position
in the input stream using a command like
\texttt{fin.position = hddm\_x.streamposition(start,offset,status)}. Until
another position assignment is executed, reading proceeds in a serial
fashion starting from the last record read from the stream.

\section{Acknowledgments}

This work is supported by the U.S. National Science Foundation under grant 1812415

\section{References}
\end{document}
