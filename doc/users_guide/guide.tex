%% ****** Start of file template.aps ****** %
%%
%%
%%   This file is part of the APS files in the REVTeX 4 distribution.
%%   Version 4.0 of REVTeX, August 2001
%%
%%
%%   Copyright (c) 2001 The American Physical Society.
%%
%%   See the REVTeX 4 README file for restrictions and more information.
%%
%
% This is a template for producing manuscripts for use with REVTEX 4.0
% Copy this file to another name and then work on that file.
% That way, you always have this original template file to use.
%
% Group addresses by affiliation; use superscriptaddress for long
% author lists, or if there are many overlapping affiliations.
% For Phys. Rev. appearance, change preprint to twocolumn.
% Choose pra, prb, prc, prd, pre, prl, prstab, or rmp for journal
%  Add 'draft' option to mark overfull boxes with black boxes
%  Add 'showpacs' option to make PACS codes appear
%  Add 'showkeys' option to make keywords appear
\documentclass{revtex4}
%\documentclass[aps,prl,preprint,superscriptaddress]{revtex4}
%\documentclass[aps,prl,twocolumn,groupedaddress]{revtex4}
\usepackage[dvipdf]{graphicx}
%\usepackage{dcolumn}

% You should use BibTeX and apsrev.bst for references
% Choosing a journal automatically selects the correct APS
% BibTeX style file (bst file), so only uncomment the line
% below if necessary.
%\bibliographystyle{apsrev}

\setlength{\textwidth}{6.5in}

\begin{document}

% Use the \preprint command to place your local institutional report
% number in the upper righthand corner of the title page in preprint mode.
% Multiple \preprint commands are allowed.
% Use the 'preprintnumbers' class option to override journal defaults
% to display numbers if necessary
%\preprint{}

%Title of paper
\title{HDDM User's Guide}

% repeat the \author .. \affiliation  etc. as needed
% \email, \thanks, \homepage, \altaffiliation all apply to the current
% author. Explanatory text should go in the []'s, actual e-mail
% address or url should go in the {}'s for \email and \homepage.
% Please use the appropriate macro foreach each type of information

% \affiliation command applies to all authors since the last
% \affiliation command. The \affiliation command should follow the
% other information
% \affiliation can be followed by \email, \homepage, \thanks as well.
%\homepage[]{Your web page}
%\thanks{}
%\altaffiliation{}
\author{R.T. Jones}
\affiliation{University of Connecticut}

%Collaboration name if desired (requires use of superscriptaddress
%option in \documentclass). \noaffiliation is required (may also be
%used with the \author command).
%\collaboration can be followed by \email, \homepage, \thanks as well.
\collaboration{GlueX}
\email{richard.t.jones@uconn.edu}
%\noaffiliation

\date{February 9, 2021}

\begin{abstract}
The HDDM (Hierarchical Document Data Model) is an xml schema for expressing
the meaning and relationships of streaming data from scientific instruments.
The design is based on a hierarchical network where each node in the graph
has a single parent node, multiple key-value attributes, and an arbitrary
number of child nodes, similar to a the elements in an xml document.
The model is adapted specifically to the case of repetitive data models
such as appear in the data stream from a high-energy physics experiment.
The representation of the model in xml is an essential feature, although
instantiation in memory does not involve the creation of explicit textual
elements or construction of a Document Object Model (DOM) for the data.
The HDDM toolkit includes tools to express HDDM streams in xml, check their
validity against the schema, and serialize/deserialize from container
objects in memory. Originally written in c, HDDM provides application
programmer interfaces for C++ and python as well. In addition to its own
native data format, applications that use HDDM to access their data can
also read/write standard HDF5 files and ROOT trees.
\end{abstract}

% insert suggested PACS numbers in braces on next line
%\pacs{}
% insert suggested keywords - APS authors don't need to do this
%\keywords{}

\setlength{\topmargin}{0in}

%\maketitle must follow title, authors, abstract, \pacs, and \keywords
\maketitle

% body of paper here - Use proper section commands
% References should be done using the \cite, \ref, and \label commands

%% The normal text is displayed in two-column format, but special
%% sections spanning both columns can be inserted within the page
%% format so that long equations can be displayed. Use
%% sparingly.
%%\begin{widetext}
%% put long equation here
%%\end{widetext}
%
%% figures should be put into the text as floats.
%% Use the graphics or graphicx packages (distributed with LaTeX2e)
%% and the \includegraphics macro defined in those packages.
%% See the LaTeX Graphics Companion by Michel Goosens, Sebastian Rahtz,
%% and Frank Mittelbach for instance.
%%
%% Here is an example of the general form of a figure:
%% Fill in the caption in the braces of the \caption{} command. Put the label
%% that you will use with \ref{} command in the braces of the \label{} command.
%% Use the figure* environment if the figure should span across the
%% entire page. There is no need to do explicit centering.
%
%%\begin{turnpage}
%% Surround figure environment with turnpage environment for landscape
%% figure
%% \begin{turnpage}
%% \begin{figure}
%% \includegraphics{}%
%% \caption{\label{}}
%% \end{figure}
%% \end{turnpage}
%
%% tables should appear as floats within the text
%%
%% Here is an example of the general form of a table:
%% Fill in the caption in the braces of the \caption{} command. Put the label
%% that you will use with \ref{} command in the braces of the \label{} command.
%% Insert the column specifiers (l, r, c, d, etc.) in the empty braces of the
%% \begin{tabular}{} command.
%% The ruledtabular enviroment adds doubled rules to table and sets a
%% reasonable default table settings.
%% Use the table* environment to get a full-width table in two-column
%% Add \usepackage{longtable} and the longtable (or longtable*}
%% environment for nicely formatted long tables. Or use the the [H]
%% placement option to break a long table (with less control than 
%% in longtable).
%
%
%% Surround table environment with turnpage environment for landscape
%% table
%% \begin{turnpage}
%% \begin{table}
%% \caption{\label{}}
%% \begin{ruledtabular}
%% \begin{tabular}{}
%% \end{tabular}
%% \end{ruledtabular}
%% \end{table}
%% \end{turnpage}
%
%% Specify following sections are appendices. Use \appendix* if there
%% only one appendix.
%%\appendix
%%\section{}
%

\section{Introduction}

The HDDM toolkit provides the scientist with a means to format streaming data
from a scientific instrument into a structured self-describing byte stream of
binary data that is platform-independent and easy to browse, filter, transform
extend, annotate, and validate using standard xml tools. The purpose of this 
User's Guide is to describe the use and operation of the HDDM tools, describing
how to define a new data model or inspect an existing one, how to create new
hddm files or read data from existing files. HDDM tools automatically generate
the object classes that represent the data described in the user's model, with
methods to access the object data in memory as well as to serialize/deserialize
themselves between memory and an external byte stream connected to an ordinary
file on disk or to a network socket. The underlying implementation of the i/o
library is in C++, so it provides good performance in terms of data rate to/from
byte streams, with optional on-{}the-{}fly compression/decompression and data 
integrity verification. In addition to oridnary sequential access to data,
random-{}access to records at an arbitrary offset in a stream is also provided
for streams that support random seeks, without the need to read the entire stream.

\section{Templates and schemas}

Every HDDM stream has an associated data model, expressed either in the form of a
standard XML schema, or more compactly, in the form of a HDDM template.  Tools are
provided to translate between the schema and HDDM template description of the model.
A HDDM template is a short xml document that describes the structure of one record
in the hddm stream. Every hddm stream has a copy of its template in plain-text
UTF-8 at the beginning, followed by a sequence of data records in a compact binary
format. The template contains all of the information necessary to reconstruct the
original data objects from the serialized records, together with their hierarchical
arrangement. A simple example of a template is given in Fig.~\ref{simple_template}.

\begin{figure}
\begin{minipage}{12cm}
\begin{verbatim}
<?xml version="1.0" encoding="UTF-8"?>
<HDDM class="x" version="1.0" xmlns="http://www.gluex.org/hddm">
  <student name="string" minOccurs="0">
    <enrolled year="int" semester="int" maxOccurs="unbounded">
      <course credits="int" title="string" maxOccurs="unbounded">
        <result grade="string" Pass="boolean" />
      </course>
    </enrolled>
  </student>
</HDDM>
\end{verbatim}
\end{minipage}
\caption{\label{simple_template}
A simple example of a HDDM data model template. The data stream would
consist of a stream of <student> records of unbounded length, each with the
same hierachical structure of contained data elements. Only the values designated
by simple types, ``int'', ``string'', etc.\ are actually stored in the byte stream.}
\end{figure}

All of the records in the file represent repeats of this basic structure, with 
different values in the data fields. All actual data values are represented as
attributes of tags. Attributes that are assigned type names (``string'', ``int'',
``long'', ``float'', ``double'', ``boolean'', ``anyURI'', and ``Particle\_t'')
are user data. Any other values assigned to attributes other than these simple
types are treated as annotations in the data model, eg.\ to specify the units
assigned to physical values, and do not take up space in the file (other than
in the template header). Some of these literal attributes function as metadata,
eg.\ you might want to add an attribute unit=``GeV'' to document the units used
for other attributes in a tag. Other special attributes like minOccurs/maxOccurs
take special values that tell the data model whether a given element is always
present in every record or may be omitted (minOccurs=``0'') or whether it may
be repeated any number of times (maxOccurs=``unbounded''), as in a standard
xml schema. The top-{}level element is special in that it must always be named
HDDM and have the attributes shown in Fig.~\ref{simple_template}. The class
attribute of the \texttt{<HDDM>} element is any (preferrably short) string that
you chose for the family of data models you are creating. Choose a short, unique
name for your class because it is used in the type names of user objects that are
defined written by the hddm user library. Its purpose is to prevents collisions
between different HDDM stream types that may coexist in a single application.

Templates provide an intuitive way of specifying the structure of a data record
in a hddm stream. For most users, this is all they need to know about in order
to define their data models. For those familiar with XML schema validation,
there is a more formal way to specify the structure of an xml document which is
called a {\em xml schema}. HDDM uses schemas in two different ways. The first
is to specify the structure of the templates themselves. The template shown in
Fig.~\ref{simple_template} conforms to a schema called \url{http://www.gluex.org/hddm}.
This is not a URL to anywhere; it is a URI known as an {\em xml namespace}, as
suggested by the name of the \texttt{xmlns} attribute in the HDDM tag of the template.
The schema for this document type is found in hddm\_schema.xsl in the schema
directory of the distribution. The second use of schemas is related to the fact
that every record in a hddm stream is a valid xml fragment corresponding to a
schema against which it can be validated. The hddm toolkit provides a pair of
tools {\em hddm-{}schema} and {\em schema-{}hddm} that convert back and forth
between templates and schema. The two are equivalent ways of representing the
same information about the structure of a hddm record, with the schema being
more complete and standards-{}based, while the template is shorter and more 
intuitive to most users. Schemas provide a much more general set of constraints
that can be expressed for the data and relationships between them, but 
experience has shown that their practical use for this purpose is limited to
special instances where standards-based data validation must be performed.
The remainder of this document deals mainly only with templates.

\section{How to get started}

The hddm toolkit is distributed as a github repository \url{https://github.com/rjones30/HDDM}.
Instructions for how to download and build HDDM are given in the INSTALL file
provided at the top level of the download tree. The hddm tools are installed
by the installation procedure into the bin directory under the installation
base. Before continuing to read this document, make sure that the basic HDDM tools
including hddm-{}xml, xml-{}hddm, hddm-{}c, hddm-{}cpp, hddm-{}py, and xml-{}xml
are in your shell PATH. These tools are not the hddm libraries themselves, but
the code generators you need to construct user-callable libraries from your 
HDDM template.

If you already have a HDDM data file that you want to read, you can generate
the i/o user library that you can use to read from it and optionally to write
a selection of the records to a new HDDM output file. The template that the
code generators need to generate the user library is present in the header of
the HDDM file that you want to read. Simply providing the data file as input
to \texttt{hddm-c} will generate c header and implementation files that you
need to include on the compiler command line together with your c application
code for your project, and similarly, \texttt{hddm-cpp} in the case of C++
applications, or \texttt{hddm-py} to generate a python module. Of the three
supported programming languages, the python implementation is the least
verbose and most readable, so it is recommended as a starting point for
someone experimenting with HDDM.

Independent of any user programs or language-{}specific API, the hddm toolkit
provides two tools that can be used to read and write hddm files directly from
the command line. The following command accepts any valid hddm file as input
and prints the contents of the file in plain-{}text xml to standard output.

\vspace{1cm}
\begin{minipage}{1cm}
\begin{verbatim}
$ hddm-xml [-n <count>] [-o <output.xml>] <datafile.hddm> [...]
\end{verbatim}
\end{minipage}
\vspace{1cm}

The reverse action is provided by the \texttt{xml-{}hddm} tool.

\vspace{1cm}
\begin{minipage}{1cm}
\begin{verbatim}
$ xml-hddm [-n <count>] -t template.xml <input.xml> [...]
\end{verbatim}
\end{minipage}
\vspace{1cm}

The full XML rendition of a data file with many records is highly verbose,
which makes the plain-text xml rendering of a HDDM stream of limited practical
interest, except to visually browse the data, or to make changes using a text
editor. The reversibility of the conversion between xml and hddm representations
can be useful in cases where one might doubt the fidelity of the encoding being
used by hddm. These two tools do not require any compile-{}and-{}link step each
time the template is changed, so they are very useful to quickly inspect the
contents of a hddm file. Keep them handy when working through the
language-{}specific procedures below.

\section{HDDM in python}

If you have access to a hddm file that was already written, copy it into your
work directory and use as a template in building a python module to read the data
into python list objects. The HDDM package distribution directory contains a 
simple example in \texttt{models/exam1x.hddm}. Use the following commands to build
the python module that you can use to read the contents of this file in the
form of python objects.

\vspace{1cm}
\begin{minipage}{12cm}
\begin{verbatim}
$ hddm-cpp exam1x.hddm  # builds the underlying C++ library
$ hddm-py exam1x.hddm   # builds the python interface
$ python setup_hddm_x.py build -b build_hddm_x # creates the module hddm_x
\end{verbatim}
\end{minipage}
\vspace{1cm}

In this example, I assigned `x' as the HDDM {\em class} letter (see the HDDM tag
in the template header). You should change it to whatever the class abbreviation
you chose for your hddm data model. The above steps should create a shared library
that starts with \texttt{hddm\_x} in your work directory. Copy that module to a
directory in your PYTHONPATH where you usually place your private python modules.

Execute the following interactive python script to print the contents of the
example hddm file in plain text.

\vspace{1cm}
\begin{minipage}{12cm}
\begin{verbatim}
import hddm_x
for rec in hddm_x.istream("examx.hddm"):
   print(rec)
\end{verbatim}
\end{minipage}
\vspace{1cm}

To see the same data printed out as a properly formatted xml document, replace
the \texttt{print(rec)} in the above python hddm reader with 
\texttt{print(rec.toXML())}'. 

\subsection{writing hddm files in python}

For this example, I continue using the same template as was used in the
example python hddm reader above. You should already have built and installed
the hddm\_x python module and installed it in your PYTHONPATH, using the build
steps listed above. Execute the following in a fresh interactive python session
to write a new output hddm file from scratch, starting only from the template
and some fake user data.

\vspace{1cm}
\begin{minipage}{12cm}
\begin{verbatim}
import hddm_x
ofs = hddm_x.ostream('examx2.hddm')
xrec = hddm_x.HDDM()
student = xrec.addStudents()
student[0].name = 'Humphrey Gaston'
enrolled = student[0].addEnrolleds()
enrolled[0].year = 2005
enrolled[0].semester = 2
course = enrolled[0].addCourses(3)
course[0].credits = 3
course[0].title = 'Beginning Russian'
result = course[0].addResults()
result[0].grade = 'A-'
result[0].Pass = True
course[1].credits = 1
course[1].title = 'Bohemian Poetry'
result = course[1].addResults()
result[0].grade = 'C'
result[0].Pass = 1
course[2].credits = 4
course[2].title = 'Developmental Psychology'
result = course[2].addResults()
result[0].grade = 'B+'
result[0].Pass = True
ofs.write(xrec)
\end{verbatim}
\end{minipage}
\vspace{1cm}

This generates a new hddm file called examx2.hddm. Now running the above 3-{}line
python hddm reader on exam2x.hddm should yield the following output.

\vspace{1cm}
\begin{minipage}{12cm}
\begin{verbatim}
HDDM
  student name="Humphrey Gaston"
    enrolled semester=2 year=2005
      course credits=3 title="Beginning Russian"
        result Pass=false grade="A-"
      course credits=1 title="Bohemian Poetry"
        result Pass=false grade="C"
      course credits=4 title="Developmental Psychology"
        result Pass=false grade="B+"
\end{verbatim}
\end{minipage}
\vspace{1cm}

The structure of the output record you are writing is already known to
the python module because it is configured with your template. All that
the writer needs to do is to fill in the elements and assign the values
of the defined attributes. You begin by creating an empty record by calling
the HDDM() default constructor. Then you populate the structure top-{}down
by calling addXXXs() methods for each tag XXX under that. The name XXXs
is the name of the tag element in the template in a capitalized-{}plural 
form. The addXXXs() methods take a single optional int argument, which
is the number of copies of that element that need to be added (default 1).
They return a list that can be indexed in the usual python fashion to give
access to the individual members of the list. Each of these has addXXXs()
methods for each of its contents, and so on down the tree. You can omit
whole branches of the tree by simply not calling the corresponding 
addXXXs() method, although xml rules require that you specify minOccurs=``0''
for the containing tag in the template if you plan to make that subtree
optional. As soon as a new element list is created, you can fill in the
values of its attributes using simple assignment semantics, as illustrated
in the example. The names of the python data members are the same as the
names of the attributes in the template.

\subsection{reading hddm files in python}

For this illustration, I assume you have created the file examx2.hddm using
the writer described in the previous section. The following python program
lets you open this file and extract bits of information from the first record,
writing a summary report at the end. Of course, in actual practice, a hddm
file would contain many records and the analysis would loop over many instances
of student.

\vspace{1cm}
\begin{minipage}{12cm}
\begin{verbatim}
import hddm_x
ifs = hddm_x.istream("examx2.hddm")
xrec = ifs.read()
total_enrolled = 0
total_courses = 0
total_credits = 0
total_passed = 0
for course in xrec.getCourses():
   total_courses += 1
   if course.getResult().Pass:
      if course.year > 1992:
         total_credits += course.credits
      total_passed += 1
   total_enrolled += 1
print(course.name, "enrolled in", total_courses, " courses",
       "and passed" , total_passed, "of them,\n",
       "earning a total of", total_credits, "credits.\n")
\end{verbatim}
\end{minipage}
\vspace{1cm}

Running the above code should produce output like the following:

\vspace{1cm}
\begin{minipage}{12cm}
\begin{verbatim}
Humphrey Gaston enrolled in 3 courses and passed 3 of them,
earning a total of 8 credits.
\end{verbatim}
\end{minipage}
\vspace{1cm}

In addition to each tag supporting the lookup (via getXXXs methods) of the
tags immediately appearing under it in the template hierarchy, the 
top-{}level HDDM record provides global getXXXs methods for every tag
throughout the hierarchy, and returns all instances of a given tag that
appear anywhere in the record, in the order of their appearance. The
istream object itself also functions as an iterable in python so the
construct, \texttt{for rec in hddm\_x.istream('exam2.hddm'):} would
look over all records in the input file, assigning the \texttt{rec}
iteration variable to each record as it is read from the input stream.
Likewise, each call to method getXXXs() returns a python list of tag
element objects that is iterable using the usual python \texttt{for}
semantics, as illustrated for xrec.getCourses() above. As before, the
individual attributes of each tag instance are accessed as plain data 
members of their host object. The standard pythonb list functions (eg.
len(list), str(list), repr(list)) all work as expected for these hddm
tag list objects returned by getXXXs() method. These natural python
iteration and accessor semantics provide a quick-{}and-{}simple
prototyping framework for analysis of repetitve experimental data.

\subsection{advanced features of the python API}

See section \ref{Advanced_features} {Advanced features} below.

%%
%%\section{HDDM in C++}
%%
%%If you have access to a hddm file that was written by someone else, copy it into your work directory and use a text editor to extract the header into a file, which you may call ``exam.xml''. Use the following commands to build the C++ module that you will need to read the contents of this file.
%%
%%
%%
%%\TemplatePreformat{$\text{ }$\newline{}
%%\${}$\text{ }${}hddm-{}cpp$\text{ }${}exam.xml$\text{ }$\newline{}
%%\${}$\text{ }${}mv$\text{ }${}hddm\_x.cpp$\text{ }${}hddm\_x++.cpp$\text{ }$\newline{}
%%\${}$\text{ }${}g++$\text{ }${}-{}c$\text{ }${}$\text{ }${}hddm\_x++.cpp$\text{ }${}XString.cpp$\text{ }${}XParsers.cpp$\text{ }${}md5.c$\text{ }${}-{}I$\text{ }$\newline{}
%%$\text{ }${}\${}HALLD\_HOME/\${}BMS\_OSNAME/include$\text{ }${}\textbackslash{}$\text{ }$\newline{}
%%-{}I\${}XERCESCROOT/include$\text{ }${}-{}L$\text{ }${}\${}XERCESCROOT/lib$\text{ }${}-{}l$\text{ }${}xerces-{}c$\text{ }${}-{}L$\text{ }$\newline{}
%%$\text{ }${}\${}HALLD\_HOME/\${}BMS\_OSNAME/lib$\text{ }${}\textbackslash{}$\text{ }$\newline{}
%%-{}lxstream$\text{ }${}-{}lz$\text{ }${}-{}lbz2$\text{ }$\newline{}
%%}
%%The rename step from hddm\_x.cpp to hddm\_x++.cpp is inserted to prevent any confusion between the files created in this section and those created below for use with the c API, but is not essential if this is the only interface that is of interest to you.
%%
%%
%%
%%\subsection{writing hddm files in C++}
%%
%%For this example, I return to the template listed at the top of this page, which I call ``exam2.xml''. Use the build steps above to build the hddm\_x C++ API object module. Create a new file and cut/paste the contents of the box below into it, then save it.
%%
%%
%%
%%\TemplatePreformat{$\text{ }$\newline{}
%%\#include$\text{ }${}{\mbox{$<$}}fstream{\mbox{$>$}}$\text{ }$\newline{}
%%\#include$\text{ }${}{\mbox{$\text{``}$}}hddm\_x.hpp{\mbox{$\text{''}$}}$\text{ }$\newline{}
%%int$\text{ }${}main()$\text{ }$\newline{}
%%\{$\text{ }$\newline{}
%%$\text{ }${}$\text{ }${}$\text{ }${}//$\text{ }${}build$\text{ }${}the$\text{ }${}nodal$\text{ }${}structure$\text{ }${}for$\text{ }${}this$\text{ }${}record$\text{ }${}and$\text{ }${}fill$\text{ }${}in$\text{ }${}its$\text{ }${}values$\text{ }$\newline{}
%%$\text{ }${}$\text{ }${}$\text{ }${}hddm\_x::HDDM$\text{ }${}xrec;$\text{ }$\newline{}
%%$\text{ }${}$\text{ }${}$\text{ }${}hddm\_x::StudentList$\text{ }${}student$\text{ }${}=$\text{ }${}xrec.addStudents();$\text{ }$\newline{}
%%$\text{ }${}$\text{ }${}$\text{ }${}student().setName({\mbox{$\text{``}$}}Humphrey$\text{ }${}Gaston{\mbox{$\text{''}$}});$\text{ }$\newline{}
%%$\text{ }${}$\text{ }${}$\text{ }${}hddm\_x::EnrolledList$\text{ }${}enrolled$\text{ }${}=$\text{ }${}student().addEnrolleds();$\text{ }$\newline{}
%%$\text{ }${}$\text{ }${}$\text{ }${}enrolled().setYear(2005);$\text{ }$\newline{}
%%$\text{ }${}$\text{ }${}$\text{ }${}enrolled().setSemester(2);$\text{ }$\newline{}
%%$\text{ }${}$\text{ }${}$\text{ }${}hddm\_x::CourseList$\text{ }${}course$\text{ }${}=$\text{ }${}enrolled().addCourses(3);$\text{ }$\newline{}
%%$\text{ }${}$\text{ }${}$\text{ }${}course(0).setCredits(3);$\text{ }$\newline{}
%%$\text{ }${}$\text{ }${}$\text{ }${}course(0).setTitle({\mbox{$\text{``}$}}Beginning$\text{ }${}Russian{\mbox{$\text{''}$}});$\text{ }$\newline{}
%%$\text{ }${}$\text{ }${}$\text{ }${}course(0).addResults();$\text{ }$\newline{}
%%$\text{ }${}$\text{ }${}$\text{ }${}course(0).getResult().setGrade({\mbox{$\text{``}$}}A-{}{\mbox{$\text{''}$}});$\text{ }$\newline{}
%%$\text{ }${}$\text{ }${}$\text{ }${}course(0).getResult().setPass(true);$\text{ }$\newline{}
%%$\text{ }${}$\text{ }${}$\text{ }${}course(1).setCredits(1);$\text{ }$\newline{}
%%$\text{ }${}$\text{ }${}$\text{ }${}course(1).setTitle({\mbox{$\text{``}$}}Bohemian$\text{ }${}Poetry{\mbox{$\text{''}$}});$\text{ }$\newline{}
%%$\text{ }${}$\text{ }${}$\text{ }${}course(1).addResults();$\text{ }$\newline{}
%%$\text{ }${}$\text{ }${}$\text{ }${}course(1).getResult().setGrade({\mbox{$\text{``}$}}C{\mbox{$\text{''}$}});$\text{ }$\newline{}
%%$\text{ }${}$\text{ }${}$\text{ }${}course(1).getResult().setPass(1);$\text{ }$\newline{}
%%$\text{ }${}$\text{ }${}$\text{ }${}course(2).setCredits(4);$\text{ }$\newline{}
%%$\text{ }${}$\text{ }${}$\text{ }${}course(2).setTitle({\mbox{$\text{``}$}}Developmental$\text{ }${}Psychology{\mbox{$\text{''}$}});$\text{ }$\newline{}
%%$\text{ }${}$\text{ }${}$\text{ }${}course(2).addResults();$\text{ }$\newline{}
%%$\text{ }${}$\text{ }${}$\text{ }${}course(2).getResult().setGrade({\mbox{$\text{``}$}}B+{\mbox{$\text{''}$}});$\text{ }$\newline{}
%%$\text{ }${}$\text{ }${}$\text{ }${}course(2).getResult().setPass(true);$\text{ }$\newline{}
%%$\text{ }${}$\text{ }$\newline{}
%%$\text{ }${}$\text{ }${}$\text{ }${}std::ofstream$\text{ }${}ofs(“exam2.hddm”);$\text{ }$\newline{}
%%$\text{ }${}$\text{ }${}$\text{ }${}hddm\_x::ostream$\text{ }${}ostr(ofs);$\text{ }$\newline{}
%%$\text{ }${}$\text{ }${}$\text{ }${}ostr$\text{ }${}{\mbox{$<$}}{\mbox{$<$}}$\text{ }${}xrec;$\text{ }$\newline{}
%%$\text{ }${}$\text{ }${}$\text{ }${}xrec.clear();$\text{ }$\newline{}
%%$\text{ }${}$\text{ }${}$\text{ }${}return$\text{ }${}0;$\text{ }$\newline{}
%%\}$\text{ }$\newline{}
%%}
%%Copy this C++ program to a file called write\_exam.cpp and compile it into an executable using a command like the following.
%%
%%
%%
%%\TemplatePreformat{$\text{ }$\newline{}
%%\${}$\text{ }${}g++$\text{ }${}-{}o$\text{ }${}write\_exam$\text{ }${}write\_exam.cpp$\text{ }${}hddm\_x++.o$\text{ }${}-{}I.$\text{ }${}-{}I$\text{ }$\newline{}
%%$\text{ }${}\${}HALLD\_HOME/\${}BMS\_OSNAME/include$\text{ }${}\textbackslash{}$\text{ }$\newline{}
%%-{}I\${}XERCESCROOT/include$\text{ }${}-{}L$\text{ }${}\${}XERCESCROOT/lib$\text{ }${}-{}l$\text{ }${}xerces-{}c$\text{ }${}-{}L$\text{ }$\newline{}
%%$\text{ }${}\${}HALLD\_HOME/\${}BMS\_OSNAME/lib$\text{ }${}\textbackslash{}$\text{ }$\newline{}
%%-{}lxstream$\text{ }${}-{}lz$\text{ }${}-{}lbz2$\text{ }$\newline{}
%%}
%%This may need to be customized for your own build environment. Once it completes successfully, you will find the executable write\_exam in the working directory. Run it as ``./write\_exam2'' and it should create a new hddm file called exam2.hddm. Running ``hddm-{}xml write\_exam2.hddm'' should produce output like the following.
%%
%%
%%
%%\TemplatePreformat{$\text{ }$\newline{}
%%HDDM$\text{ }$\newline{}
%%$\text{ }${}$\text{ }${}student$\text{ }${}name={\mbox{$\text{``}$}}Humphrey$\text{ }${}Gaston{\mbox{$\text{''}$}}$\text{ }$\newline{}
%%$\text{ }${}$\text{ }${}$\text{ }${}$\text{ }${}enrolled$\text{ }${}semester=2$\text{ }${}year=2005$\text{ }$\newline{}
%%$\text{ }${}$\text{ }${}$\text{ }${}$\text{ }${}$\text{ }${}$\text{ }${}course$\text{ }${}credits=3$\text{ }${}title={\mbox{$\text{``}$}}Beginning$\text{ }${}Russian{\mbox{$\text{''}$}}$\text{ }$\newline{}
%%$\text{ }${}$\text{ }${}$\text{ }${}$\text{ }${}$\text{ }${}$\text{ }${}$\text{ }${}$\text{ }${}result$\text{ }${}Pass=false$\text{ }${}grade={\mbox{$\text{``}$}}A-{}{\mbox{$\text{''}$}}$\text{ }$\newline{}
%%$\text{ }${}$\text{ }${}$\text{ }${}$\text{ }${}$\text{ }${}$\text{ }${}course$\text{ }${}credits=1$\text{ }${}title={\mbox{$\text{``}$}}Bohemian$\text{ }${}Poetry{\mbox{$\text{''}$}}$\text{ }$\newline{}
%%$\text{ }${}$\text{ }${}$\text{ }${}$\text{ }${}$\text{ }${}$\text{ }${}$\text{ }${}$\text{ }${}result$\text{ }${}Pass=false$\text{ }${}grade={\mbox{$\text{``}$}}C{\mbox{$\text{''}$}}$\text{ }$\newline{}
%%$\text{ }${}$\text{ }${}$\text{ }${}$\text{ }${}$\text{ }${}$\text{ }${}course$\text{ }${}credits=4$\text{ }${}title={\mbox{$\text{``}$}}Developmental$\text{ }${}Psychology{\mbox{$\text{''}$}}$\text{ }$\newline{}
%%$\text{ }${}$\text{ }${}$\text{ }${}$\text{ }${}$\text{ }${}$\text{ }${}$\text{ }${}$\text{ }${}result$\text{ }${}Pass=false$\text{ }${}grade={\mbox{$\text{``}$}}B+{\mbox{$\text{''}$}}$\text{ }$\newline{}
%%}
%%The structure of the output record you are writing is already known to the program because it knows about your template. All that you need to do is to fill in the elements and assign the values of the attributes. You begin by creating an empty record by calling the HDDM() default constructor. Then you populate the structure top-{}down by calling addXXXs() methods for each tag XXX under that. The name XXXs is the name of the tag element in the template in a capitalized-{}plural form. The addXXXs() methods take a single optional int argument, which is the number of copies of that element that need to be added (default is 1). They return a subclass of std::list that can be iterated over in the usual fashion, or indexed with operator()(int) to access the individual members of the list. Each of these has addXXXs() methods for each of its contents, and so on down the tree. You can omit whole branches of the tree by simply not calling the corresponding addXXXs() method, although xml rules require that you specify minOccurs=``0'' for the containing tag in the template if you plan to do that. As soon as a new element list is created, you can fill in the values of its attributes using set{\mbox{$<$}}attname{\mbox{$>$}} methods, as illustrated in the example, where {\mbox{$<$}}attname{\mbox{$>$}} is a capitalized version of the names of the attribute in the template.
%%
%%
%%
%%\subsection{reading hddm files in C++}
%%
%%For this illustration, I assume you have created the file exam2.hddm using the instructions in the previous section. The following C++ program lets you open this file and extract bits of information from the first record, writing a summary report at the end. Of course, in actual practice a hddm file would contain many records and the analysis would loop over many instances student.
%%
%%
%%
%%\TemplatePreformat{$\text{ }$\newline{}
%%\#include$\text{ }${}{\mbox{$<$}}fstream{\mbox{$>$}}$\text{ }$\newline{}
%%\#include$\text{ }${}{\mbox{$\text{``}$}}hddm\_x.hpp{\mbox{$\text{''}$}}$\text{ }$\newline{}
%%int$\text{ }${}main()$\text{ }$\newline{}
%%\{$\text{ }$\newline{}
%%$\text{ }${}$\text{ }${}$\text{ }${}std::ifstream$\text{ }${}ifs({\mbox{$\text{``}$}}exam2.hddm{\mbox{$\text{''}$}});$\text{ }$\newline{}
%%$\text{ }${}$\text{ }${}$\text{ }${}hddm\_x::HDDM$\text{ }${}xrec;$\text{ }$\newline{}
%%$\text{ }${}$\text{ }${}$\text{ }${}hddm\_x::istream$\text{ }${}istr(ifs);$\text{ }$\newline{}
%%$\text{ }${}$\text{ }${}$\text{ }${}istr$\text{ }${}{\mbox{$>$}}{\mbox{$>$}}$\text{ }${}xrec;$\text{ }$\newline{}
%%$\text{ }${}$\text{ }${}$\text{ }${}hddm\_x::CourseList$\text{ }${}course$\text{ }${}=$\text{ }${}xrec.getCourses();$\text{ }$\newline{}
%%$\text{ }${}$\text{ }${}$\text{ }${}int$\text{ }${}total\_courses$\text{ }${}=course.size();$\text{ }$\newline{}
%%$\text{ }${}$\text{ }${}$\text{ }${}int$\text{ }${}total\_enrolled$\text{ }${}=$\text{ }${}0;$\text{ }$\newline{}
%%$\text{ }${}$\text{ }${}$\text{ }${}int$\text{ }${}total\_credits$\text{ }${}=$\text{ }${}0;$\text{ }$\newline{}
%%$\text{ }${}$\text{ }${}$\text{ }${}int$\text{ }${}total\_passed$\text{ }${}=$\text{ }${}0;$\text{ }$\newline{}
%%$\text{ }${}$\text{ }${}$\text{ }${}hddm\_x::CourseList::iterator$\text{ }${}iter;$\text{ }$\newline{}
%%$\text{ }${}$\text{ }${}$\text{ }${}for$\text{ }${}(iter$\text{ }${}=$\text{ }${}course.begin();$\text{ }${}iter{\mbox{$~$}}!=$\text{ }${}course.end();$\text{ }${}++iter)$\text{ }${}\{$\text{ }$\newline{}
%%$\text{ }${}$\text{ }${}$\text{ }${}$\text{ }${}$\text{ }${}$\text{ }${}if$\text{ }${}(iter-{}{\mbox{$>$}}getResult().getPass())$\text{ }${}\{$\text{ }$\newline{}
%%$\text{ }${}$\text{ }${}$\text{ }${}$\text{ }${}$\text{ }${}$\text{ }${}$\text{ }${}$\text{ }${}$\text{ }${}if$\text{ }${}(iter-{}{\mbox{$>$}}getYear()$\text{ }${}{\mbox{$>$}}$\text{ }${}1992)$\text{ }${}\{$\text{ }$\newline{}
%%$\text{ }${}$\text{ }${}$\text{ }${}$\text{ }${}$\text{ }${}$\text{ }${}$\text{ }${}$\text{ }${}$\text{ }${}$\text{ }${}$\text{ }${}$\text{ }${}total\_credits$\text{ }${}+=$\text{ }${}iter-{}{\mbox{$>$}}getCredits();$\text{ }$\newline{}
%%$\text{ }${}$\text{ }${}$\text{ }${}$\text{ }${}$\text{ }${}$\text{ }${}$\text{ }${}$\text{ }${}$\text{ }${}\}$\text{ }$\newline{}
%%$\text{ }${}$\text{ }${}$\text{ }${}$\text{ }${}$\text{ }${}$\text{ }${}$\text{ }${}$\text{ }${}$\text{ }${}++total\_passed;$\text{ }$\newline{}
%%$\text{ }${}$\text{ }${}$\text{ }${}$\text{ }${}$\text{ }${}$\text{ }${}\}$\text{ }$\newline{}
%%$\text{ }${}$\text{ }${}$\text{ }${}\}$\text{ }$\newline{}
%%$\text{ }${}$\text{ }${}$\text{ }${}std::cout$\text{ }${}{\mbox{$<$}}{\mbox{$<$}}$\text{ }${}course().getName()$\text{ }${}{\mbox{$<$}}{\mbox{$<$}}$\text{ }${}{\mbox{$\text{``}$}}$\text{ }${}enrolled$\text{ }${}in$\text{ }${}{\mbox{$\text{''}$}}$\text{ }$\newline{}
%%$\text{ }${}$\text{ }${}$\text{ }${}$\text{ }${}$\text{ }${}$\text{ }${}$\text{ }${}$\text{ }${}$\text{ }${}$\text{ }${}$\text{ }${}$\text{ }${}$\text{ }${}{\mbox{$<$}}{\mbox{$<$}}$\text{ }${}total\_courses$\text{ }${}{\mbox{$<$}}{\mbox{$<$}}$\text{ }${}{\mbox{$\text{``}$}}$\text{ }${}courses$\text{ }${}{\mbox{$\text{''}$}}$\text{ }$\newline{}
%%$\text{ }${}$\text{ }${}$\text{ }${}$\text{ }${}$\text{ }${}$\text{ }${}$\text{ }${}$\text{ }${}$\text{ }${}$\text{ }${}$\text{ }${}$\text{ }${}$\text{ }${}{\mbox{$<$}}{\mbox{$<$}}$\text{ }${}{\mbox{$\text{``}$}}and$\text{ }${}passed$\text{ }${}{\mbox{$\text{''}$}}$\text{ }${}{\mbox{$<$}}{\mbox{$<$}}$\text{ }${}total\_passed$\text{ }${}{\mbox{$<$}}{\mbox{$<$}}$\text{ }${}{\mbox{$\text{``}$}}$\text{ }${}of$\text{ }${}them,$\text{ }${}{\mbox{$\text{''}$}}$\text{ }${}{\mbox{$<$}}{\mbox{$<$}}$\text{ }${}std::endl$\text{ }$\newline{}
%%$\text{ }${}$\text{ }${}$\text{ }${}$\text{ }${}$\text{ }${}$\text{ }${}$\text{ }${}$\text{ }${}$\text{ }${}$\text{ }${}$\text{ }${}$\text{ }${}$\text{ }${}{\mbox{$<$}}{\mbox{$<$}}$\text{ }${}{\mbox{$\text{``}$}}earning$\text{ }${}a$\text{ }${}total$\text{ }${}of$\text{ }${}{\mbox{$\text{''}$}}$\text{ }${}{\mbox{$<$}}{\mbox{$<$}}$\text{ }${}total\_credits$\text{ }$\newline{}
%%$\text{ }${}$\text{ }${}$\text{ }${}$\text{ }${}$\text{ }${}$\text{ }${}$\text{ }${}$\text{ }${}$\text{ }${}$\text{ }${}$\text{ }${}$\text{ }${}$\text{ }${}{\mbox{$<$}}{\mbox{$<$}}$\text{ }${}{\mbox{$\text{``}$}}$\text{ }${}credits.{\mbox{$\text{''}$}}$\text{ }${}{\mbox{$<$}}{\mbox{$<$}}$\text{ }${}std::endl;$\text{ }$\newline{}
%%$\text{ }${}$\text{ }${}$\text{ }${}return$\text{ }${}0;$\text{ }$\newline{}
%%\}$\text{ }$\newline{}
%%}
%%Running the above code should produce output like the following:
%%
%%
%%
%%\TemplatePreformat{$\text{ }$\newline{}
%%Humphrey$\text{ }${}Gaston$\text{ }${}enrolled$\text{ }${}in$\text{ }${}3$\text{ }${}courses$\text{ }${}and$\text{ }${}passed$\text{ }${}3$\text{ }${}of$\text{ }${}them,$\text{ }$\newline{}
%%earning$\text{ }${}a$\text{ }${}total$\text{ }${}of$\text{ }${}8$\text{ }${}credits.$\text{ }$\newline{}
%%}
%%The HDDM tag list objects are std:list containers so they support STL forward iterator semantics. In addition to each tag supporting the lookup (via getXXXs methods) of the tags immediately appearing under it in the template hierarchy, the top-{}level HDDM record provides global getXXXs methods for every tag throughout the hierarchy, and returns all instances of a given tag that appear anywhere in the record, listed in the order of their appearance. Each call to a method getXXXs() returns a C++ list of tag element objects. The individual attributes of each tag instance are accessed using the get{\mbox{$<$}}Attname{\mbox{$>$}} members of their host object. The standard list methods (eg. size(), begin(), end()) all work as expected for these tag list objects returned by getXXXs() methods. If you are not sure how to do something, a quick browse through the header file should give a good overview of the capabilities of these classes.
%%
%%
%%
%%\subsection{advanced features of the C++ API}
%%
%%See section on \myhref{\#Advanced_features}{Advanced features} below.
%%
%%
%%
%%\section{HDDM in c}
%%
%%If you have access to a hddm file that was written by someone else, copy it into your work directory and use a text editor to extract the template header into a file, which you may call ``exam.xml''. Use the following commands to build the c library that you will need to read the contents of this file.
%%
%%
%%
%%\TemplatePreformat{$\text{ }$\newline{}
%%\${}$\text{ }${}hddm-{}c$\text{ }${}exam.xml$\text{ }$\newline{}
%%\${}$\text{ }${}g++$\text{ }${}-{}c$\text{ }${}$\text{ }${}hddm\_x.c$\text{ }${}XString.cpp$\text{ }${}XParsers.cpp$\text{ }${}md5.c$\text{ }${}-{}I$\text{ }$\newline{}
%%$\text{ }${}\${}HALLD\_HOME/\${}BMS\_OSNAME/include$\text{ }${}\textbackslash{}$\text{ }$\newline{}
%%-{}I\${}XERCESCROOT/include$\text{ }${}-{}L$\text{ }${}\${}XERCESCROOT/lib$\text{ }${}-{}l$\text{ }${}xerces-{}c$\text{ }$\newline{}
%%}
%%\subsection{writing hddm files in c}
%%
%%For this example, I return to the template listed at the top of this page, which I call ``exam2.xml''. Use the build steps above to build the hddm\_x c API object module. Create a new file and cut/paste the contents of the box below into it, then save it.
%%
%%
%%
%%\TemplatePreformat{$\text{ }$\newline{}
%%\#include$\text{ }${}{\mbox{$\text{``}$}}hddm\_x.h{\mbox{$\text{''}$}}$\text{ }$\newline{}
%%$\text{ }${}$\text{ }$\newline{}
%%int$\text{ }${}main()$\text{ }$\newline{}
%%\{$\text{ }$\newline{}
%%$\text{ }${}$\text{ }${}$\text{ }${}x\_iostream\_t*$\text{ }${}fp;$\text{ }$\newline{}
%%$\text{ }${}$\text{ }${}$\text{ }${}x\_HDDM\_t*$\text{ }${}exam2;$\text{ }$\newline{}
%%$\text{ }${}$\text{ }${}$\text{ }${}x\_Student\_t*$\text{ }${}$\text{ }${}student;$\text{ }$\newline{}
%%$\text{ }${}$\text{ }${}$\text{ }${}x\_Enrolleds\_t*$\text{ }${}enrolleds;$\text{ }$\newline{}
%%$\text{ }${}$\text{ }${}$\text{ }${}x\_Courses\_t*$\text{ }${}courses;$\text{ }$\newline{}
%%$\text{ }${}$\text{ }${}$\text{ }${}x\_Result\_t*$\text{ }${}result;$\text{ }$\newline{}
%%$\text{ }${}$\text{ }${}$\text{ }${}string\_t$\text{ }${}name;$\text{ }$\newline{}
%%$\text{ }${}$\text{ }${}$\text{ }${}string\_t$\text{ }${}grade;$\text{ }$\newline{}
%%$\text{ }${}$\text{ }${}$\text{ }${}string\_t$\text{ }${}course;$\text{ }$\newline{}
%%$\text{ }${}$\text{ }$\newline{}
%%$\text{ }${}$\text{ }${}$\text{ }${}//$\text{ }${}first$\text{ }${}build$\text{ }${}the$\text{ }${}complete$\text{ }${}nodal$\text{ }${}structure$\text{ }${}for$\text{ }${}this$\text{ }${}record$\text{ }$\newline{}
%%$\text{ }${}$\text{ }${}$\text{ }${}exam2$\text{ }${}=$\text{ }${}make\_x\_HDDM();$\text{ }$\newline{}
%%$\text{ }${}$\text{ }${}$\text{ }${}exam2-{}{\mbox{$>$}}student$\text{ }${}=$\text{ }${}student$\text{ }${}=$\text{ }${}make\_x\_Student();$\text{ }$\newline{}
%%$\text{ }${}$\text{ }${}$\text{ }${}student-{}{\mbox{$>$}}enrolleds$\text{ }${}=$\text{ }${}enrolleds$\text{ }${}=$\text{ }${}make\_x\_Enrolleds(99);$\text{ }$\newline{}
%%$\text{ }${}$\text{ }${}$\text{ }${}enrolleds-{}{\mbox{$>$}}mult$\text{ }${}=$\text{ }${}1;$\text{ }$\newline{}
%%$\text{ }${}$\text{ }${}$\text{ }${}enrolleds-{}{\mbox{$>$}}in{$\text{[}$}0{$\text{]}$}.courses$\text{ }${}=$\text{ }${}courses$\text{ }${}=$\text{ }${}make\_x\_Courses(99);$\text{ }$\newline{}
%%$\text{ }${}$\text{ }${}$\text{ }${}courses-{}{\mbox{$>$}}mult$\text{ }${}=$\text{ }${}3;$\text{ }$\newline{}
%%$\text{ }${}$\text{ }${}$\text{ }${}courses-{}{\mbox{$>$}}in{$\text{[}$}0{$\text{]}$}.result$\text{ }${}=$\text{ }${}make\_x\_Result();$\text{ }$\newline{}
%%$\text{ }${}$\text{ }${}$\text{ }${}courses-{}{\mbox{$>$}}in{$\text{[}$}1{$\text{]}$}.result$\text{ }${}=$\text{ }${}make\_x\_Result();$\text{ }$\newline{}
%%$\text{ }${}$\text{ }${}$\text{ }${}courses-{}{\mbox{$>$}}in{$\text{[}$}2{$\text{]}$}.result$\text{ }${}=$\text{ }${}make\_x\_Result();$\text{ }$\newline{}
%%$\text{ }${}$\text{ }$\newline{}
%%$\text{ }${}$\text{ }${}$\text{ }${}//$\text{ }${}now$\text{ }${}fill$\text{ }${}in$\text{ }${}the$\text{ }${}attribute$\text{ }${}data$\text{ }${}for$\text{ }${}this$\text{ }${}record$\text{ }$\newline{}
%%$\text{ }${}$\text{ }${}$\text{ }${}name$\text{ }${}=$\text{ }${}malloc(30);$\text{ }$\newline{}
%%$\text{ }${}$\text{ }${}$\text{ }${}strcpy(name,{\mbox{$\text{``}$}}Humphrey$\text{ }${}Gaston{\mbox{$\text{''}$}});$\text{ }$\newline{}
%%$\text{ }${}$\text{ }${}$\text{ }${}student-{}{\mbox{$>$}}name$\text{ }${}=$\text{ }${}name;$\text{ }$\newline{}
%%$\text{ }${}$\text{ }${}$\text{ }${}enrolleds-{}{\mbox{$>$}}in{$\text{[}$}0{$\text{]}$}.year$\text{ }${}=$\text{ }${}2005;$\text{ }$\newline{}
%%$\text{ }${}$\text{ }${}$\text{ }${}enrolleds-{}{\mbox{$>$}}in{$\text{[}$}0{$\text{]}$}.semester$\text{ }${}=$\text{ }${}2;$\text{ }$\newline{}
%%$\text{ }${}$\text{ }${}$\text{ }${}courses-{}{\mbox{$>$}}in{$\text{[}$}0{$\text{]}$}.credits$\text{ }${}=$\text{ }${}3;$\text{ }$\newline{}
%%$\text{ }${}$\text{ }${}$\text{ }${}course$\text{ }${}=$\text{ }${}malloc(30);$\text{ }$\newline{}
%%$\text{ }${}$\text{ }${}$\text{ }${}courses-{}{\mbox{$>$}}in{$\text{[}$}0{$\text{]}$}.title$\text{ }${}=$\text{ }${}strcpy(course,{\mbox{$\text{``}$}}Beginning$\text{ }${}Russian{\mbox{$\text{''}$}});$\text{ }$\newline{}
%%$\text{ }${}$\text{ }${}$\text{ }${}grade$\text{ }${}=$\text{ }${}malloc(5);$\text{ }$\newline{}
%%$\text{ }${}$\text{ }${}$\text{ }${}courses-{}{\mbox{$>$}}in{$\text{[}$}0{$\text{]}$}.result-{}{\mbox{$>$}}grade$\text{ }${}=$\text{ }${}strcpy(grade,{\mbox{$\text{``}$}}A-{}{\mbox{$\text{''}$}});$\text{ }$\newline{}
%%$\text{ }${}$\text{ }${}$\text{ }${}courses-{}{\mbox{$>$}}in{$\text{[}$}0{$\text{]}$}.result-{}{\mbox{$>$}}Pass$\text{ }${}=$\text{ }${}1;$\text{ }$\newline{}
%%$\text{ }${}$\text{ }${}$\text{ }${}courses-{}{\mbox{$>$}}in{$\text{[}$}1{$\text{]}$}.credits$\text{ }${}=$\text{ }${}1;$\text{ }$\newline{}
%%$\text{ }${}$\text{ }${}$\text{ }${}course$\text{ }${}=$\text{ }${}malloc(30);$\text{ }$\newline{}
%%$\text{ }${}$\text{ }${}$\text{ }${}courses-{}{\mbox{$>$}}in{$\text{[}$}1{$\text{]}$}.title$\text{ }${}=$\text{ }${}strcpy(course,{\mbox{$\text{``}$}}Bohemian$\text{ }${}Poetry{\mbox{$\text{''}$}});$\text{ }$\newline{}
%%$\text{ }${}$\text{ }${}$\text{ }${}grade$\text{ }${}=$\text{ }${}malloc(5);$\text{ }$\newline{}
%%$\text{ }${}$\text{ }${}$\text{ }${}courses-{}{\mbox{$>$}}in{$\text{[}$}1{$\text{]}$}.result-{}{\mbox{$>$}}grade$\text{ }${}=$\text{ }${}strcpy(grade,{\mbox{$\text{``}$}}C{\mbox{$\text{''}$}});$\text{ }$\newline{}
%%$\text{ }${}$\text{ }${}$\text{ }${}courses-{}{\mbox{$>$}}in{$\text{[}$}1{$\text{]}$}.result-{}{\mbox{$>$}}Pass$\text{ }${}=$\text{ }${}1;$\text{ }$\newline{}
%%$\text{ }${}$\text{ }${}$\text{ }${}courses-{}{\mbox{$>$}}in{$\text{[}$}2{$\text{]}$}.credits$\text{ }${}=$\text{ }${}4;$\text{ }$\newline{}
%%$\text{ }${}$\text{ }${}$\text{ }${}course$\text{ }${}=$\text{ }${}malloc(30);$\text{ }$\newline{}
%%$\text{ }${}$\text{ }${}$\text{ }${}courses-{}{\mbox{$>$}}in{$\text{[}$}2{$\text{]}$}.title$\text{ }${}=$\text{ }${}strcpy(course,{\mbox{$\text{``}$}}Developmental$\text{ }${}Psychology{\mbox{$\text{''}$}});$\text{ }$\newline{}
%%$\text{ }${}$\text{ }${}$\text{ }${}grade$\text{ }${}=$\text{ }${}malloc(5);$\text{ }$\newline{}
%%$\text{ }${}$\text{ }${}$\text{ }${}courses-{}{\mbox{$>$}}in{$\text{[}$}2{$\text{]}$}.result-{}{\mbox{$>$}}grade$\text{ }${}=$\text{ }${}strcpy(grade,{\mbox{$\text{``}$}}B+{\mbox{$\text{''}$}});$\text{ }$\newline{}
%%$\text{ }${}$\text{ }${}$\text{ }${}courses-{}{\mbox{$>$}}in{$\text{[}$}2{$\text{]}$}.result-{}{\mbox{$>$}}Pass$\text{ }${}=$\text{ }${}1;$\text{ }$\newline{}
%%$\text{ }${}$\text{ }$\newline{}
%%$\text{ }${}$\text{ }${}$\text{ }${}//$\text{ }${}now$\text{ }${}open$\text{ }${}a$\text{ }${}file$\text{ }${}and$\text{ }${}write$\text{ }${}this$\text{ }${}one$\text{ }${}record$\text{ }${}into$\text{ }${}it$\text{ }$\newline{}
%%$\text{ }${}$\text{ }${}$\text{ }${}fp$\text{ }${}=$\text{ }${}init\_x\_HDDM({\mbox{$\text{``}$}}exam2.hddm{\mbox{$\text{''}$}});$\text{ }$\newline{}
%%$\text{ }${}$\text{ }${}$\text{ }${}flush\_x\_HDDM(exam2,fp);$\text{ }$\newline{}
%%$\text{ }${}$\text{ }${}$\text{ }${}close\_x\_HDDM(fp);$\text{ }$\newline{}
%%$\text{ }${}$\text{ }$\newline{}
%%$\text{ }${}$\text{ }${}$\text{ }${}return$\text{ }${}0;$\text{ }$\newline{}
%%\}$\text{ }$\newline{}
%%}
%%Copy this c program to a file called write\_exam2.c and compile it into an executable using a command like the following.
%%
%%
%%
%%\TemplatePreformat{$\text{ }$\newline{}
%%\${}$\text{ }${}gcc$\text{ }${}-{}o$\text{ }${}write\_exam2$\text{ }${}write\_exam2.c$\text{ }${}hddm\_x.o$\text{ }${}-{}I.$\text{ }${}-{}I$\text{ }$\newline{}
%%$\text{ }${}\${}HALLD\_HOME/\${}BMS\_OSNAME/include$\text{ }${}\textbackslash{}$\text{ }$\newline{}
%%-{}I\${}XERCESCROOT/include$\text{ }${}-{}L$\text{ }${}\${}XERCESCROOT/lib$\text{ }${}-{}l$\text{ }${}xerces-{}c$\text{ }$\newline{}
%%}
%%This may need to be customized for your own build environment. Once it completes successfully, you will find the executable write\_exam2 in the working directory. Run it as ``./write\_exam2'' and it should create a new hddm file called exam2.hddm. Running ``hddm-{}xml write\_exam2.hddm'' should produce output like the following.
%%
%%
%%
%%\TemplatePreformat{$\text{ }$\newline{}
%%HDDM$\text{ }$\newline{}
%%$\text{ }${}$\text{ }${}student$\text{ }${}name={\mbox{$\text{``}$}}Humphrey$\text{ }${}Gaston{\mbox{$\text{''}$}}$\text{ }$\newline{}
%%$\text{ }${}$\text{ }${}$\text{ }${}$\text{ }${}enrolled$\text{ }${}semester=2$\text{ }${}year=2005$\text{ }$\newline{}
%%$\text{ }${}$\text{ }${}$\text{ }${}$\text{ }${}$\text{ }${}$\text{ }${}course$\text{ }${}credits=3$\text{ }${}title={\mbox{$\text{``}$}}Beginning$\text{ }${}Russian{\mbox{$\text{''}$}}$\text{ }$\newline{}
%%$\text{ }${}$\text{ }${}$\text{ }${}$\text{ }${}$\text{ }${}$\text{ }${}$\text{ }${}$\text{ }${}result$\text{ }${}Pass=false$\text{ }${}grade={\mbox{$\text{``}$}}A-{}{\mbox{$\text{''}$}}$\text{ }$\newline{}
%%$\text{ }${}$\text{ }${}$\text{ }${}$\text{ }${}$\text{ }${}$\text{ }${}course$\text{ }${}credits=1$\text{ }${}title={\mbox{$\text{``}$}}Bohemian$\text{ }${}Poetry{\mbox{$\text{''}$}}$\text{ }$\newline{}
%%$\text{ }${}$\text{ }${}$\text{ }${}$\text{ }${}$\text{ }${}$\text{ }${}$\text{ }${}$\text{ }${}result$\text{ }${}Pass=false$\text{ }${}grade={\mbox{$\text{``}$}}C{\mbox{$\text{''}$}}$\text{ }$\newline{}
%%$\text{ }${}$\text{ }${}$\text{ }${}$\text{ }${}$\text{ }${}$\text{ }${}course$\text{ }${}credits=4$\text{ }${}title={\mbox{$\text{``}$}}Developmental$\text{ }${}Psychology{\mbox{$\text{''}$}}$\text{ }$\newline{}
%%$\text{ }${}$\text{ }${}$\text{ }${}$\text{ }${}$\text{ }${}$\text{ }${}$\text{ }${}$\text{ }${}result$\text{ }${}Pass=false$\text{ }${}grade={\mbox{$\text{``}$}}B+{\mbox{$\text{''}$}}$\text{ }$\newline{}
%%}
%%This example explains most of what you need to know to set up hddm structures in memory, and then write them to an output file. All storage for hddm data is allocated on the heap. Most of this allocation is carried out automatically by the make\_x\_XXXs() functions, although for strings (char arrays) the user needs to allocate initial storage for the values himself. Memory pointed to by the pointers returned by the make\_x\_XXXs() functions is owned by the user code until the pointer to it gets assigned to a hddm struct member that is designated to hold it. After that, the memory is owned by the top-{}level HDDM struct, and should only be freed by calling the flush\_x\_HDDM() method. Calling flush\_x\_HDDM(record, fp) with its second argument (FILE*) open to an output file causes the record to be written to the output file. Calling it as flush\_x\_HDDM(record,0) causes it to bypass the output step. Either way, flush\_x\_HDDM() frees all memory owned by the HDDM record, discarding its contents, before it returns.
%%
%%
%%The structure of the output record you are writing is already known to the program because it knows about your template. All that you need to do is to fill in the elements and assign the values of the attributes. You begin by creating an empty record by calling make\_x\_HDDM(). Then you populate the structure top-{}down by calling make\_x\_XXXs() for each tag XXX and assigning pointers to each one into the appropriate structure element of the parent element. The name XXXs is the name of the tag element in the template in a capitalized-{}plural form. The addXXXs() methods take a single optional int argument, which is the number of copies of that element that need to be added (default is 1). They return a pointer to an array of struct pointers which can be indexed in the usual c-{}fashion to access the individual members of the array. Each of the contained elements within a given host tag have a corresponding pointer in the host struct that must be assigned in the user code to the value returned by the make\_x\_XXXs() function, as illustrated. Any such pointers that are not assigned remain null (initialized by make\_x\_XXXs) and represent parts of the template tree that are missing from the record. This is a perfectly valid hddm record, but it must be checked for by the code that reads the record since c has no automatic checking of the validity of pointers during dereferencing. As soon as a new struct array element is created, you can fill in the values of its attribute members using direct assignment semantics, as illustrated in the example above. Any values that are not explicitly assigned remain at the default values, typically zero or null.
%%
%%
%%
%%\subsection{reading hddm files in c}
%%
%%For this illustration, I assume you have created the file exam2.hddm using the instructions in the previous section. The following c program lets you open this file and extract bits of information from the first record, writing a summary report at the end. Of course, in actual practice a hddm file would contain many records and the analysis would loop over many instances student.
%%
%%
%%
%%\TemplatePreformat{$\text{ }$\newline{}
%%\#include$\text{ }${}{\mbox{$\text{``}$}}hddm\_x.h{\mbox{$\text{''}$}}$\text{ }$\newline{}
%%$\text{ }${}$\text{ }$\newline{}
%%int$\text{ }${}main()$\text{ }$\newline{}
%%\{$\text{ }$\newline{}
%%$\text{ }${}$\text{ }${}$\text{ }${}x\_iostream\_t*$\text{ }${}fp;$\text{ }$\newline{}
%%$\text{ }${}$\text{ }${}$\text{ }${}x\_HDDM\_t*$\text{ }${}exam2;$\text{ }$\newline{}
%%$\text{ }${}$\text{ }${}$\text{ }${}x\_Student\_t*$\text{ }${}student;$\text{ }$\newline{}
%%$\text{ }${}$\text{ }${}$\text{ }${}x\_Enrolleds\_t*$\text{ }${}enrolleds;$\text{ }$\newline{}
%%$\text{ }${}$\text{ }${}$\text{ }${}int$\text{ }${}enrolled;$\text{ }$\newline{}
%%$\text{ }${}$\text{ }${}$\text{ }${}x\_Courses\_t*$\text{ }${}courses;$\text{ }$\newline{}
%%$\text{ }${}$\text{ }${}$\text{ }${}int$\text{ }${}course;$\text{ }$\newline{}
%%$\text{ }${}$\text{ }${}$\text{ }${}int$\text{ }${}total\_enrolled,total\_courses,total\_credits,total\_passed;$\text{ }$\newline{}
%%$\text{ }${}$\text{ }$\newline{}
%%$\text{ }${}//$\text{ }${}read$\text{ }${}a$\text{ }${}record$\text{ }${}from$\text{ }${}the$\text{ }${}file$\text{ }$\newline{}
%%$\text{ }${}$\text{ }${}$\text{ }${}fp$\text{ }${}=$\text{ }${}open\_x\_HDDM({\mbox{$\text{``}$}}exam2.hddm{\mbox{$\text{''}$}});$\text{ }$\newline{}
%%$\text{ }${}$\text{ }${}$\text{ }${}if$\text{ }${}(fp$\text{ }${}==$\text{ }${}NULL)$\text{ }${}\{$\text{ }$\newline{}
%%${\text{ }}${}${\text{ }}${}${\text{ }}${}${\text{ }}${}$\text{ }${}$\text{ }${}printf({\mbox{$\text{``}$}}Error$\text{ }${}-{}$\text{ }${}could$\text{ }${}not$\text{ }${}open$\text{ }${}input$\text{ }${}file$\text{ }${}exam2.hddm\textbackslash{}n{\mbox{$\text{''}$}});$\text{ }$\newline{}
%%${\text{ }}${}${\text{ }}${}${\text{ }}${}${\text{ }}${}$\text{ }${}$\text{ }${}exit(1);$\text{ }$\newline{}
%%$\text{ }${}$\text{ }${}$\text{ }${}\}$\text{ }$\newline{}
%%$\text{ }${}$\text{ }${}$\text{ }${}exam2$\text{ }${}=$\text{ }${}read\_x\_HDDM(fp);$\text{ }$\newline{}
%%$\text{ }${}$\text{ }${}$\text{ }${}if$\text{ }${}(exam2$\text{ }${}==$\text{ }${}NULL)$\text{ }${}\{$\text{ }$\newline{}
%%${\text{ }}${}${\text{ }}${}${\text{ }}${}${\text{ }}${}$\text{ }${}$\text{ }${}printf({\mbox{$\text{``}$}}End$\text{ }${}of$\text{ }${}file$\text{ }${}encountered$\text{ }${}in$\text{ }${}hddm$\text{ }${}file$\text{ }${}exam2.hddm,$\text{ }${}quitting!\textbackslash{}n{\mbox{$\text{''}$}});$\text{ }$\newline{}
%%${\text{ }}${}${\text{ }}${}${\text{ }}${}${\text{ }}${}$\text{ }${}$\text{ }${}exit(2);$\text{ }$\newline{}
%%$\text{ }${}$\text{ }${}$\text{ }${}\}$\text{ }$\newline{}
%%$\text{ }${}$\text{ }$\newline{}
%%$\text{ }${}$\text{ }${}$\text{ }${}//$\text{ }${}examine$\text{ }${}the$\text{ }${}data$\text{ }${}in$\text{ }${}this$\text{ }${}record$\text{ }${}and$\text{ }${}print$\text{ }${}a$\text{ }${}summary$\text{ }$\newline{}
%%$\text{ }${}$\text{ }${}$\text{ }${}total\_enrolled$\text{ }${}=$\text{ }${}0;$\text{ }$\newline{}
%%$\text{ }${}$\text{ }${}$\text{ }${}total\_courses$\text{ }${}=$\text{ }${}0;$\text{ }$\newline{}
%%$\text{ }${}$\text{ }${}$\text{ }${}total\_credits$\text{ }${}=$\text{ }${}0;$\text{ }$\newline{}
%%$\text{ }${}$\text{ }${}$\text{ }${}total\_passed$\text{ }${}=$\text{ }${}0;$\text{ }$\newline{}
%%$\text{ }${}$\text{ }${}$\text{ }${}student$\text{ }${}=$\text{ }${}exam2-{}{\mbox{$>$}}student;$\text{ }$\newline{}
%%$\text{ }${}$\text{ }${}$\text{ }${}enrolleds$\text{ }${}=$\text{ }${}student-{}{\mbox{$>$}}enrolleds;$\text{ }$\newline{}
%%$\text{ }${}$\text{ }${}$\text{ }${}total\_enrolled$\text{ }${}=$\text{ }${}enrolleds-{}{\mbox{$>$}}mult;$\text{ }$\newline{}
%%$\text{ }${}$\text{ }${}$\text{ }${}for$\text{ }${}(enrolled=0;$\text{ }${}enrolled{\mbox{$<$}}total\_enrolled;$\text{ }${}++enrolled)$\text{ }${}\{$\text{ }$\newline{}
%%${\text{ }}${}${\text{ }}${}${\text{ }}${}${\text{ }}${}$\text{ }${}$\text{ }${}courses$\text{ }${}=$\text{ }${}enrolleds-{}{\mbox{$>$}}in{$\text{[}$}enrolled{$\text{]}$}.courses;$\text{ }$\newline{}
%%${\text{ }}${}${\text{ }}${}${\text{ }}${}${\text{ }}${}$\text{ }${}$\text{ }${}total\_courses$\text{ }${}+=$\text{ }${}courses-{}{\mbox{$>$}}mult;$\text{ }$\newline{}
%%${\text{ }}${}${\text{ }}${}${\text{ }}${}${\text{ }}${}$\text{ }${}$\text{ }${}for$\text{ }${}(course=0;$\text{ }${}course{\mbox{$<$}}courses-{}{\mbox{$>$}}mult;$\text{ }${}course++)$\text{ }${}\{$\text{ }$\newline{}
%%${\text{ }}${}${\text{ }}${}${\text{ }}${}${\text{ }}${}$\text{ }${}$\text{ }${}$\text{ }${}$\text{ }${}$\text{ }${}if$\text{ }${}(courses-{}{\mbox{$>$}}in{$\text{[}$}course{$\text{]}$}.result-{}{\mbox{$>$}}Pass)$\text{ }${}\{$\text{ }$\newline{}
%%$\text{ }${}${\text{ }}${}${\text{ }}${}${\text{ }}${}${\text{ }}${}$\text{ }${}$\text{ }${}$\text{ }${}$\text{ }${}$\text{ }${}$\text{ }${}$\text{ }${}if$\text{ }${}(enrolleds-{}{\mbox{$>$}}in{$\text{[}$}enrolled{$\text{]}$}.year$\text{ }${}{\mbox{$>$}}$\text{ }${}1992)$\text{ }${}\{$\text{ }$\newline{}
%%${\text{ }}${}${\text{ }}${}${\text{ }}${}${\text{ }}${}$\text{ }${}$\text{ }${}$\text{ }${}$\text{ }${}$\text{ }${}$\text{ }${}$\text{ }${}$\text{ }${}$\text{ }${}$\text{ }${}$\text{ }${}total\_credits$\text{ }${}+=$\text{ }${}courses-{}{\mbox{$>$}}in{$\text{[}$}course{$\text{]}$}.credits;$\text{ }$\newline{}
%%${\text{ }}${}${\text{ }}${}${\text{ }}${}${\text{ }}${}$\text{ }${}$\text{ }${}$\text{ }${}$\text{ }${}$\text{ }${}$\text{ }${}$\text{ }${}$\text{ }${}\}$\text{ }$\newline{}
%%${\text{ }}${}${\text{ }}${}${\text{ }}${}${\text{ }}${}$\text{ }${}$\text{ }${}$\text{ }${}$\text{ }${}$\text{ }${}$\text{ }${}$\text{ }${}$\text{ }${}++total\_passed;$\text{ }$\newline{}
%%${\text{ }}${}${\text{ }}${}${\text{ }}${}${\text{ }}${}$\text{ }${}$\text{ }${}$\text{ }${}$\text{ }${}$\text{ }${}\}$\text{ }$\newline{}
%%${\text{ }}${}${\text{ }}${}${\text{ }}${}${\text{ }}${}$\text{ }${}$\text{ }${}\}$\text{ }$\newline{}
%%$\text{ }${}$\text{ }${}$\text{ }${}\}$\text{ }$\newline{}
%%$\text{ }${}$\text{ }${}$\text{ }${}printf({\mbox{$\text{``}$}}\%s$\text{ }${}enrolled$\text{ }${}in{\mbox{$~$}}\%d$\text{ }${}courses.\textbackslash{}n{\mbox{$\text{''}$}},student-{}{\mbox{$>$}}name,total\_courses);$\text{ }$\newline{}
%%$\text{ }${}$\text{ }${}$\text{ }${}printf({\mbox{$\text{``}$}}He$\text{ }${}passed{\mbox{$~$}}\%d$\text{ }${}of$\text{ }${}them,$\text{ }${}earning$\text{ }${}a$\text{ }${}total$\text{ }${}of{\mbox{$~$}}\%d$\text{ }$\newline{}
%%$\text{ }${}credits.\textbackslash{}n{\mbox{$\text{''}$}},total\_passed,total\_credits);$\text{ }$\newline{}
%%$\text{ }${}$\text{ }$\newline{}
%%$\text{ }${}$\text{ }${}$\text{ }${}flush\_x\_HDDM(exam2,0);$\text{ }${}$\text{ }${}//$\text{ }${}don\textquotesingle{}t$\text{ }${}do$\text{ }${}this$\text{ }${}until$\text{ }${}you$\text{ }${}are$\text{ }${}done$\text{ }${}with$\text{ }${}exam2$\text{ }$\newline{}
%%$\text{ }${}$\text{ }${}$\text{ }${}close\_x\_HDDM(fp);$\text{ }$\newline{}
%%$\text{ }${}$\text{ }${}$\text{ }${}return$\text{ }${}0;$\text{ }$\newline{}
%%\}$\text{ }$\newline{}
%%}
%%Running the above code should produce output like the following:
%%
%%
%%
%%\TemplatePreformat{$\text{ }$\newline{}
%%Humphrey$\text{ }${}Gaston$\text{ }${}enrolled$\text{ }${}in$\text{ }${}3$\text{ }${}courses$\text{ }${}and$\text{ }${}passed$\text{ }${}3$\text{ }${}of$\text{ }${}them,$\text{ }$\newline{}
%%earning$\text{ }${}a$\text{ }${}total$\text{ }${}of$\text{ }${}8$\text{ }${}credits.$\text{ }$\newline{}
%%}
%%Having read the section above on how to write hddm records using the c interface, it should be easy to understand the meaning of the above code. The read\_x\_HDDM() call allocates all of the memory needed to stand up the full record hierarchy in memory. The flush\_x\_HDDM(record,0) call at the end of the loop ensures that all of this memory gets recycled to the heap before the next record is read in. Accessing leaf elements that are deep inside the hddm template hierarchy can only be reached by traversing all of the nodes in the tree above, which makes a simple data mining operation somewhat verbose, although it still scales well because of the hierarchical model (not like a linked list). If you are unsure about how to do something, browsing through the header file is probably not going to be easy because all of the internal functionality of the logic that supports the serialization/deserialization of the data is exposed there. The API is fairly simple: Access to the attributes themselves is through direct struct member access. Only the make\_x\_XXXs functions and the input/output functions (open, close, read, flush, skip) should be called by the user, all the rest are for internal use. As for all of the other API\textquotesingle{}s, the template itself should be the main documentation needed to write code that interacts with hddm files.
%%
%%
%%
%%\subsection{advanced features of the c API}
%%
%%The c API is no longer in active development, so it is supported only for legacy applications that rely on it. The features described in the \myhref{\#Advanced_features}{Advanced features} section below are not available using the c API. The only things that are ensured by the ongoing support of the c API is that it can files based on any valid hddm template, and that hddm files written using it can be read by applications built using any of the other API\textquotesingle{}s. The converse is not true. If an input file is not readable by the c-{}API, it prints a polite error message reporting this fact and quietly exits.
%%
%%
%%
%%\section{Advanced features}
%%
%%\subsection{on-{}the-{}fly compression/decompresson}
%%
%%HDDM streams added support for on-{}the-{}fly compression on output (and decompression on input) with the introduction of the C++ API. Because the python API is a thin wrapper around the C++ classes, it also supports this feature. Compression can obviously only be controlled when the stream is being written. It can be switched on and off at any time after the stream is opened, either before the first record is written or any time thereafter. Whenever it is turned on or off, a small marker is written to the file that tells the reader when to enable/disable decompression on the input stream. These transitions occur silently during input; no user action is needed. Two compression options are supported.
%%
%%
%%
%%
%%\begin{myenumerate}\item{} {\bfseries \allowbreak{}\setmainfont{cmunbx.ttf}[Path=/usr/share/fonts/truetype/cmu/,UprightFont=cmunrm,BoldFont=cmunbx,ItalicFont=cmunti,BoldItalicFont=cmunbi]\setmonofont{cmunbx.ttf}[Path=/usr/share/fonts/truetype/cmu/,UprightFont=cmuntt,BoldFont=cmuntb,ItalicFont=cmunit,BoldItalicFont=cmuntx]\bfseries bzip2 compression}{$\text{ }$}\allowbreak{}\setmainfont{cmunrm.ttf}[Path=/usr/share/fonts/truetype/cmu/,UprightFont=cmunrm,BoldFont=cmunbx,ItalicFont=cmunti,BoldItalicFont=cmunbi]\setmonofont{cmunrm.ttf}[Path=/usr/share/fonts/truetype/cmu/,UprightFont=cmuntt,BoldFont=cmuntb,ItalicFont=cmunit,BoldItalicFont=cmuntx] -{} This option offers the best compression ratio, a factor of about 2.5 for GlueX Monte Carlo data. It is also the most expensive in terms of cpu time needed during output. Cpu demand for decompression is much lower, more than an order of magnitude.
%%\item{} {\bfseries \allowbreak{}\setmainfont{cmunbx.ttf}[Path=/usr/share/fonts/truetype/cmu/,UprightFont=cmunrm,BoldFont=cmunbx,ItalicFont=cmunti,BoldItalicFont=cmunbi]\setmonofont{cmunbx.ttf}[Path=/usr/share/fonts/truetype/cmu/,UprightFont=cmuntt,BoldFont=cmuntb,ItalicFont=cmunit,BoldItalicFont=cmuntx]\bfseries zlib compression}{$\text{ }$}\allowbreak{}\setmainfont{cmunrm.ttf}[Path=/usr/share/fonts/truetype/cmu/,UprightFont=cmunrm,BoldFont=cmunbx,ItalicFont=cmunti,BoldItalicFont=cmunbi]\setmonofont{cmunrm.ttf}[Path=/usr/share/fonts/truetype/cmu/,UprightFont=cmuntt,BoldFont=cmuntb,ItalicFont=cmunit,BoldItalicFont=cmuntx] -{} This option offers somewhat lower compression ratios, a factor of about 1.9 for GlueX Monte Carlo data, but it is also much less expensive in terms of cpu time than bzip2, by more than a factor 3. Cpu demand for decompression is much lower than compression, as is usually the case with codecs.
%%\end{myenumerate}
%%
%%Both options are provided because each has its strengths and weaknesses in terms of cost/performance. Another factor to take into consideration when deciding which compression algorithm to use, if at all, is the implications of the compression block size on the efficiency for random access to records in the stream (for more about random access, see below). If the stream is uncompressed, random access to a particular record generates a read starting at the beginning of that specific record and only taking in the contents of that record. If the stream is compressed, the entire compression block containing the record must be decompressed before the data for the desired record can be pulled in. The compression blocks for bzip2 compression are almost 1MB in size, whereas the zlib blocks are much smaller, around 32KB. There is no general choice between these three options that will be best in every situation. The one producing the data should consider what the most likely scenarios are for reading the data, and weight the costs and benefits of compression before simply opting for bzip2.
%%
%%
%%In the C++ API, the hddm namespaces have defined the following constants to distinguish different states of compression:
%%
%%
%%
%%
%%\begin{myitemize}\item{}k\_no\_compression
%%\item{}k\_z\_compression
%%\item{}k\_bz2\_compression
%%\end{myitemize}
%%
%%One of these three values should be passed as mode to the setCompression(mode) method of the hddm\_x::ostream class to change the current state of the output stream. Only records written after this method is called will reflect this change. The present compression mode of either an input or output hddm stream can be queried by calling method getCompression(). The return value (int) can be compared with the three constants above to determine which of the three modes is presently enabled.
%%
%%
%%The python istream and ostream objects support the same interface by exposing read/write data member ``compression''. The named constants listed above are defined within the hddm\_x module namespace. Setting bz2 compression on an open ostream might look like, ``fout.compression = hddm\_x.k\_bz2\_compression''.
%%
%%
%%
%%\subsection{on-{}the-{}fly data integrity checks}
%%
%%HDDM streams added support for on-{}the-{}fly data integrity checks with the introduction of the C++ API. Because the python API is a thin wrapper around the C++ classes, it also supports this feature. Data integrity verification works by the writer computing a hash function on each output record and storing it as part of the output stream, which the reader then pulls off the stream and uses to verify that it matches what the same hash algorithm returns when applied to the data record read from the stream. Two 32-{}bit hash algorithms are currently supported by hddm.
%%
%%
%%
%%
%%\begin{myenumerate}\item{} {\bfseries \allowbreak{}\setmainfont{cmunbx.ttf}[Path=/usr/share/fonts/truetype/cmu/,UprightFont=cmunrm,BoldFont=cmunbx,ItalicFont=cmunti,BoldItalicFont=cmunbi]\setmonofont{cmunbx.ttf}[Path=/usr/share/fonts/truetype/cmu/,UprightFont=cmuntt,BoldFont=cmuntb,ItalicFont=cmunit,BoldItalicFont=cmuntx]\bfseries CRC32}{$\text{ }$}\allowbreak{}\setmainfont{cmunrm.ttf}[Path=/usr/share/fonts/truetype/cmu/,UprightFont=cmunrm,BoldFont=cmunbx,ItalicFont=cmunti,BoldItalicFont=cmunbi]\setmonofont{cmunrm.ttf}[Path=/usr/share/fonts/truetype/cmu/,UprightFont=cmuntt,BoldFont=cmuntb,ItalicFont=cmunit,BoldItalicFont=cmuntx] -{} the 32-{}bit cyclic redundancy check algorithm
%%\item{} {\bfseries \allowbreak{}\setmainfont{cmunbx.ttf}[Path=/usr/share/fonts/truetype/cmu/,UprightFont=cmunrm,BoldFont=cmunbx,ItalicFont=cmunti,BoldItalicFont=cmunbi]\setmonofont{cmunbx.ttf}[Path=/usr/share/fonts/truetype/cmu/,UprightFont=cmuntt,BoldFont=cmuntb,ItalicFont=cmunit,BoldItalicFont=cmuntx]\bfseries MD5}{$\text{ }$}\allowbreak{}\setmainfont{cmunrm.ttf}[Path=/usr/share/fonts/truetype/cmu/,UprightFont=cmunrm,BoldFont=cmunbx,ItalicFont=cmunti,BoldItalicFont=cmunbi]\setmonofont{cmunrm.ttf}[Path=/usr/share/fonts/truetype/cmu/,UprightFont=cmuntt,BoldFont=cmuntb,ItalicFont=cmunit,BoldItalicFont=cmuntx] -{} the MD5 one-{}way hash algorithm
%%\end{myenumerate}
%%
%%CRC is considered in cryptographic circles as an error detection algorithm, meaning that a single bit change in the data record will be reflected in a different value for this 32-{}bit code, and it is very difficult to get errors that cancel out and generate the same crc. This probably all we need for our data, and it is much faster to compute than MD5. MD5 is called a one-{}way hash in cryptographic jargon, which means that a single bit change in the data record will be reflected in a {\itshape \allowbreak{}\setmainfont{cmunti.ttf}[Path=/usr/share/fonts/truetype/cmu/,UprightFont=cmunrm,BoldFont=cmunbx,ItalicFont=cmunti,BoldItalicFont=cmunbi]\setmonofont{cmunti.ttf}[Path=/usr/share/fonts/truetype/cmu/,UprightFont=cmuntt,BoldFont=cmuntb,ItalicFont=cmunit,BoldItalicFont=cmuntx]\itshape vastly}{$\text{ }$}\allowbreak{}\setmainfont{cmunrm.ttf}[Path=/usr/share/fonts/truetype/cmu/,UprightFont=cmunrm,BoldFont=cmunbx,ItalicFont=cmunti,BoldItalicFont=cmunbi]\setmonofont{cmunrm.ttf}[Path=/usr/share/fonts/truetype/cmu/,UprightFont=cmuntt,BoldFont=cmuntb,ItalicFont=cmunit,BoldItalicFont=cmuntx] different value for this 32-{}bit code, with approximately 50\% of the bits changing in the hash as a result of a single bit-{}flip in the input. I suppose one might consider this marginally better for error detection, but it is more expensive to compute. Neither MD5 nor CRC32 options result in any noticeable overhead in the context of standard GlueX data streams.
%%
%%
%%In the C++ API, the hddm namespaces have defined the following constants to distinguish different states of data integrity checking:
%%
%%
%%
%%
%%\begin{myitemize}\item{}k\_no\_integrity
%%\item{}k\_crc32\_integrity
%%\item{}k\_md5\_integrity
%%\end{myitemize}
%%
%%One of these three values should be passed as mode to the setIntegrityChecks(mode) method of the hddm\_x::ostream class to change the current state of the output stream. Only records written after this method is called will reflect this change. The present integrity checking mode of either an input or output hddm stream can be queried by calling method getIntegrityChecks(). The return value (int) can be compared with the three constants above to determine which of the three modes is presently enabled.
%%
%%
%%The python istream and ostream objects support the same interface by exposing read/write data member ``integrity''. The named constants listed above are defined within the hddm\_x module namespace. Setting CRC32 compression on an open ostream might look like, ``fout.integrity = hddm\_x.k\_crc32\_integrity''.
%%
%%
%%
%%\subsection{random access to hddm records}
%%
%%HDDM streams added support for random access on input with the introduction of the python API. Because the python API is a thin wrapper around the C++ classes, it is also supported by the C++ API. Random-{}access writing to hddm streams is not supported; the access point for output streams is always after the end of the previous output record. However, the stream position pointer of the stream right before any given record was written can be recorded by the process writing the data, and then provided later to a read process which might use it to seek out particular records of interest and read only those. One can also record the position at any time when one is reading from an input stream, and then use it later to return to the same position. Random access to individual records in the input hddm stream can take place in any order, and involve displacements either forward or reverse from the position of last access. 
%%
%%
%%Stream positions can only be recorded/set at the start of records. Attempts to access a stream at a random position, or at a position that was generated by another hddm stream, will result in unpredictable behavior, most likely a segmentation fault. The following three integer values are needed to define a stream position.
%%
%%
%%
%%
%%\begin{myenumerate}\item{} {\bfseries \allowbreak{}\setmainfont{cmunbx.ttf}[Path=/usr/share/fonts/truetype/cmu/,UprightFont=cmunrm,BoldFont=cmunbx,ItalicFont=cmunti,BoldItalicFont=cmunbi]\setmonofont{cmunbx.ttf}[Path=/usr/share/fonts/truetype/cmu/,UprightFont=cmuntt,BoldFont=cmuntb,ItalicFont=cmunit,BoldItalicFont=cmuntx]\bfseries block\_start}{$\text{ }$}\allowbreak{}\setmainfont{cmunrm.ttf}[Path=/usr/share/fonts/truetype/cmu/,UprightFont=cmunrm,BoldFont=cmunbx,ItalicFont=cmunti,BoldItalicFont=cmunbi]\setmonofont{cmunrm.ttf}[Path=/usr/share/fonts/truetype/cmu/,UprightFont=cmuntt,BoldFont=cmuntb,ItalicFont=cmunit,BoldItalicFont=cmuntx] (uint64\_t) -{} absolute stream position (std::streampos value) of the beginning of the block containing the record
%%\item{} {\bfseries \allowbreak{}\setmainfont{cmunbx.ttf}[Path=/usr/share/fonts/truetype/cmu/,UprightFont=cmunrm,BoldFont=cmunbx,ItalicFont=cmunti,BoldItalicFont=cmunbi]\setmonofont{cmunbx.ttf}[Path=/usr/share/fonts/truetype/cmu/,UprightFont=cmuntt,BoldFont=cmuntb,ItalicFont=cmunit,BoldItalicFont=cmuntx]\bfseries block\_offset}{$\text{ }$}\allowbreak{}\setmainfont{cmunrm.ttf}[Path=/usr/share/fonts/truetype/cmu/,UprightFont=cmunrm,BoldFont=cmunbx,ItalicFont=cmunti,BoldItalicFont=cmunbi]\setmonofont{cmunrm.ttf}[Path=/usr/share/fonts/truetype/cmu/,UprightFont=cmuntt,BoldFont=cmuntb,ItalicFont=cmunit,BoldItalicFont=cmuntx] (uint32\_t) -{} offset with the block to the start of the designated record, or 0 if compression is disabled
%%\item{} {\bfseries \allowbreak{}\setmainfont{cmunbx.ttf}[Path=/usr/share/fonts/truetype/cmu/,UprightFont=cmunrm,BoldFont=cmunbx,ItalicFont=cmunti,BoldItalicFont=cmunbi]\setmonofont{cmunbx.ttf}[Path=/usr/share/fonts/truetype/cmu/,UprightFont=cmuntt,BoldFont=cmuntb,ItalicFont=cmunit,BoldItalicFont=cmuntx]\bfseries block\_status}{$\text{ }$}\allowbreak{}\setmainfont{cmunrm.ttf}[Path=/usr/share/fonts/truetype/cmu/,UprightFont=cmunrm,BoldFont=cmunbx,ItalicFont=cmunti,BoldItalicFont=cmunbi]\setmonofont{cmunrm.ttf}[Path=/usr/share/fonts/truetype/cmu/,UprightFont=cmuntt,BoldFont=cmuntb,ItalicFont=cmunit,BoldItalicFont=cmuntx] (uint32\_t) -{} complete state (compression, integrity, other information about the stream state at this position)
%%\end{myenumerate}
%%
%%If a database were used to store these values, a minimum field width of 128 bits would be needed. Of course, you might want to save the name and creation date of the input file that the positions apply to, so that you don\textquotesingle{}t accidentally try to apply them to a different file than they were created for. If the stream is uncompressed then block\_offset=0, but still block\_start and block\_status would be needed. The block\_status value is typically the same for all positions in a given file or dataset, so in most cases only a single value for that variable needs to be kept together with a list of the starts and offsets for a given file that might comprise a reduced dataset.
%%
%%
%%The object class hddm\_x::streamposition is used to hold stream position information. Public data members with the names listed above are exposed for members of the streamposition class. Both hddm\_x::istream and hddm\_x::ostream classes have getPosition() members that return a streamposition value. It can either be recorded by saving the values of its three data members, or by keeping the object in memory and passing it to the corresponding istream::setPosition(streamposition) method called on an istream that is (presumably) open for input on the same file. If the 3 values are stored, they can later be quickly turned back into a streamposition object using the constructor streamposition(start,offset,status).
%%
%%
%%HDDM files that were written since this feature was introduced are marked with the capability to support random access. To check if a given file that has been opened for input on a hddm\_x::istream supports random access, simply call method getPosition() within a try-{}catch block and catch any RuntimeError that is thrown as indicating that the file does not support this feature.
%%
%%
%%Support in the python API for random access follows closely the scheme described above for C++. The hddm\_x.istream and hddm\_x.ostream classes both have read/write data members called ``position'' that reference objects of type hddm\_x.streamposition. These objects can be saved and then later assigned to an hddm\_x.istream opened on the same file to seek to the same position in the input stream using a command like ``fin.position = hddm\_x.streamposition(start,offset,status)''. Until another position assignment is executed, reading proceeds in a serial fashion starting from the last record read from the stream.
%%
%%
%%
%%\section{References}
%%
%%{$\text{[}$}1{$\text{]}$} \myhref{https://halldweb.jlab.org/doc-public/DocDB/ShowDocument?docid=65}{HDDM ``Hall D Data Model'', R.T. Jones, Gluex-{}doc-{}065, September 23, 2003.}
%%
%%
%%
%%
%%
%%										
%%										
%%				
%%			
%%		
%%		
%%			\section{Navigation menu}
%%
%%					
%%			\subsection{Views}
%%
%%
%%			
%%				
%%\begin{myitemize}
%%				\item{}\myhref{https://halldweb.jlab.org/wiki/index.php/HDDM_Programmer\%27s_Interface}{Page}
%%				\item{}\myhref{https://halldweb.jlab.org/wiki/index.php?title=Talk:HDDM_Programmer\%27s_Interface\&action=edit\&redlink=1}{Discussion}
%%				\item{}\myhref{https://halldweb.jlab.org/wiki/index.php?title=HDDM_Programmer\%27s_Interface\&action=edit}{View source}
%%				\item{}\myhref{https://halldweb.jlab.org/wiki/index.php?title=HDDM_Programmer\%27s_Interface\&action=history}{History}
%%				
%%\end{myitemize}
%%
%%							
%%		
%%				
%%				\subsection{Personal tools}
%%
%%
%%				
%%					
%%\begin{myitemize}
%%													\item{}\myhref{https://halldweb.jlab.org/wiki/index.php?title=Special:UserLogin\&returnto=HDDM+Programmer\%27s+Interface}{Log in}
%%											
%%\end{myitemize}
%%
%%				
%%			
%%			
%%				\myhref{https://halldweb.jlab.org/wiki/index.php/Main_Page}{}
%%			
%%				
%%		\subsection{Navigation}
%%
%%		
%%							
%%\begin{myitemize}
%%											\item{}\myhref{https://halldweb.jlab.org/wiki/index.php/Main_Page}{Main page}
%%											\item{}\myhref{https://halldweb.jlab.org/wiki/index.php/Special:RecentChanges}{Recent changes}
%%											\item{}\myhref{https://halldweb.jlab.org/wiki/index.php/Special:Random}{Random page}
%%											\item{}\myhref{https://www.mediawiki.org/wiki/Special:MyLanguage/Help:Contents}{Help}
%%									
%%\end{myitemize}
%%
%%					
%%		
%%			
%%			\subsection{Search}
%%
%%
%%			
%%				
%%					
%%					
%%					{\mbox{$~$}}
%%						
%%				
%%
%%							
%%		
%%			
%%			\subsection{Tools}
%%
%%
%%			
%%				
%%\begin{myitemize}
%%											\item{}\myhref{https://halldweb.jlab.org/wiki/index.php/Special:WhatLinksHere/HDDM_Programmer\%27s_Interface}{What links here}
%%											\item{}\myhref{https://halldweb.jlab.org/wiki/index.php/Special:RecentChangesLinked/HDDM_Programmer\%27s_Interface}{Related changes}
%%											\item{}\myhref{https://halldweb.jlab.org/wiki/index.php/Special:SpecialPages}{Special pages}
%%											\item{}\myhref{https://halldweb.jlab.org/wiki/index.php?title=HDDM_Programmer\%27s_Interface\&printable=yes}{Printable version}
%%											\item{}\myhref{https://halldweb.jlab.org/wiki/index.php?title=HDDM_Programmer\%27s_Interface\&oldid=80206}{Permanent link}
%%											\item{}\myhref{https://halldweb.jlab.org/wiki/index.php?title=HDDM_Programmer\%27s_Interface\&action=info}{Page information}
%%					\item{}\myhref{https://halldweb.jlab.org/wiki/index.php?title=Special:CiteThisPage\&page=HDDM_Programmer\%27s_Interface\&id=80206}{Cite this page}				
%%\end{myitemize}
%%
%%							
%%		
%%			
%%		
%%					
%%		
%%		
%%
%%\chapter{Contributors}
%%\label{Contributors}
%%\begin{longtable}{rp{0.6\linewidth}}
%%\textbf{Edits}&\textbf{User}\\
%%73& \myhref{https://halldweb.jlab.org/wiki/index.php\%3ftitle=User:Jonesrt\&action=edit\&redlink=1}{Jonesrt}\\
%%2& \myhref{https://halldweb.jlab.org/wiki/index.php/User:Marki}{Marki}\\
%%\end{longtable}
%%\pagebreak
%%\listoffigures
%%\label{ListOfFigures}
%%\begin{itemize}
%%\item GFDL: Gnu Free Documentation License. \url{http://www.gnu.org/licenses/fdl.html}
%%\item cc-by-sa-3.0:  Creative Commons Attribution ShareAlike 3.0 License. \url{http://creativecommons.org/licenses/by-sa/3.0/} 
%%\item cc-by-sa-2.5:  Creative Commons Attribution ShareAlike 2.5 License. \url{http://creativecommons.org/licenses/by-sa/2.5/} 
%%\item cc-by-sa-2.0:  Creative Commons Attribution ShareAlike 2.0 License. \url{http://creativecommons.org/licenses/by-sa/2.0/} 
%%\item cc-by-sa-1.0:  Creative Commons Attribution ShareAlike 1.0 License. \url{http://creativecommons.org/licenses/by-sa/1.0/} 
%%\item cc-by-2.0:  Creative Commons Attribution 2.0 License.  \url{http://creativecommons.org/licenses/by/2.0/}  
%%\item cc-by-2.0:  Creative Commons Attribution 2.0 License. \url{http://creativecommons.org/licenses/by/2.0/deed.en}  
%%\item cc-by-2.5:  Creative Commons Attribution 2.5 License. \url{http://creativecommons.org/licenses/by/2.5/deed.en}  
%%\item cc-by-3.0:  Creative Commons Attribution 3.0 License. \url{http://creativecommons.org/licenses/by/3.0/deed.en}  
%%\item GPL:  GNU General Public License. \url{http://www.gnu.org/licenses/gpl-2.0.txt} 
%%\item LGPL:  GNU Lesser General Public License. \url{http://www.gnu.org/licenses/lgpl.html}
%% \item PD: This image is in the public domain.
%%\item ATTR:  The copyright holder of this file allows anyone to use it for any purpose, provided that the copyright holder is properly attributed. Redistribution, derivative work, commercial use, and all other use is permitted. 
%%\item EURO: This is the common (reverse) face of a euro coin. The copyright on the design of the common face of the euro coins belongs to the European Commission. Authorised is reproduction in a format without relief (drawings, paintings, films) provided they are not detrimental to the image of the euro.
%%\item LFK: Lizenz Freie Kunst. \url{http://artlibre.org/licence/lal/de} 
%%\item CFR: Copyright free use. 
%%\item EPL: Eclipse Public License. \url{http://www.eclipse.org/org/documents/epl-v10.php} 
%%\end{itemize}
%%Copies of the GPL, the LGPL as well as a GFDL are included in chapter \mylref{Licenses}{Licenses}. Please note that images in the public domain do not require attribution. You may click on the image numbers in the following table to open the webpage of the images in your webbrower.
%%\pagebreak
%%\small
%%\begin{longtable}{|p{0.05\textwidth}|p{0.6\textwidth}|p{0.15\textwidth}|}
%%\hline
%%
%%\end{longtable}
%%\pagebreak\pagebreak
%%
%%\printindex
%%
%%\KOMAoptions{fontsize=9pt,DIV=90,BCOR=0pt} 
%%\pagebreak
%%\chapter{Licenses}
%%\label{Licenses}
%%{\tiny
%%\section {GNU GENERAL PUBLIC LICENSE}
%%\begin{multicols}{4}
%%
%%Version 3, 29 June 2007
%%
%%Copyright © 2007 Free Software Foundation, Inc. <http://fsf.org/>
%%
%%Everyone is permitted to copy and distribute verbatim copies of this license document, but changing it is not allowed.
%%Preamble
%%
%%The GNU General Public License is a free, copyleft license for software and other kinds of works.
%%
%%The licenses for most software and other practical works are designed to take away your freedom to share and change the works. By contrast, the GNU General Public License is intended to guarantee your freedom to share and change all versions of a program--to make sure it remains free software for all its users. We, the Free Software Foundation, use the GNU General Public License for most of our software; it applies also to any other work released this way by its authors. You can apply it to your programs, too.
%%
%%When we speak of free software, we are referring to freedom, not price. Our General Public Licenses are designed to make sure that you have the freedom to distribute copies of free software (and charge for them if you wish), that you receive source code or can get it if you want it, that you can change the software or use pieces of it in new free programs, and that you know you can do these things.
%%
%%To protect your rights, we need to prevent others from denying you these rights or asking you to surrender the rights. Therefore, you have certain responsibilities if you distribute copies of the software, or if you modify it: responsibilities to respect the freedom of others.
%%
%%For example, if you distribute copies of such a program, whether gratis or for a fee, you must pass on to the recipients the same freedoms that you received. You must make sure that they, too, receive or can get the source code. And you must show them these terms so they know their rights.
%%
%%Developers that use the GNU GPL protect your rights with two steps: (1) assert copyright on the software, and (2) offer you this License giving you legal permission to copy, distribute and/or modify it.
%%
%%For the developers' and authors' protection, the GPL clearly explains that there is no warranty for this free software. For both users' and authors' sake, the GPL requires that modified versions be marked as changed, so that their problems will not be attributed erroneously to authors of previous versions.
%%
%%Some devices are designed to deny users access to install or run modified versions of the software inside them, although the manufacturer can do so. This is fundamentally incompatible with the aim of protecting users' freedom to change the software. The systematic pattern of such abuse occurs in the area of products for individuals to use, which is precisely where it is most unacceptable. Therefore, we have designed this version of the GPL to prohibit the practice for those products. If such problems arise substantially in other domains, we stand ready to extend this provision to those domains in future versions of the GPL, as needed to protect the freedom of users.
%%
%%Finally, every program is threatened constantly by software patents. States should not allow patents to restrict development and use of software on general-purpose computers, but in those that do, we wish to avoid the special danger that patents applied to a free program could make it effectively proprietary. To prevent this, the GPL assures that patents cannot be used to render the program non-free.
%%
%%The precise terms and conditions for copying, distribution and modification follow.
%%TERMS AND CONDITIONS
%%0. Definitions.
%%
%%“This License” refers to version 3 of the GNU General Public License.
%%
%%“Copyright” also means copyright-like laws that apply to other kinds of works, such as semiconductor masks.
%%
%%“The Program” refers to any copyrightable work licensed under this License. Each licensee is addressed as “you”. “Licensees” and “recipients” may be individuals or organizations.
%%
%%To “modify” a work means to copy from or adapt all or part of the work in a fashion requiring copyright permission, other than the making of an exact copy. The resulting work is called a “modified version” of the earlier work or a work “based on” the earlier work.
%%
%%A “covered work” means either the unmodified Program or a work based on the Program.
%%
%%To “propagate” a work means to do anything with it that, without permission, would make you directly or secondarily liable for infringement under applicable copyright law, except executing it on a computer or modifying a private copy. Propagation includes copying, distribution (with or without modification), making available to the public, and in some countries other activities as well.
%%
%%To “convey” a work means any kind of propagation that enables other parties to make or receive copies. Mere interaction with a user through a computer network, with no transfer of a copy, is not conveying.
%%
%%An interactive user interface displays “Appropriate Legal Notices” to the extent that it includes a convenient and prominently visible feature that (1) displays an appropriate copyright notice, and (2) tells the user that there is no warranty for the work (except to the extent that warranties are provided), that licensees may convey the work under this License, and how to view a copy of this License. If the interface presents a list of user commands or options, such as a menu, a prominent item in the list meets this criterion.
%%1. Source Code.
%%
%%The “source code” for a work means the preferred form of the work for making modifications to it. “Object code” means any non-source form of a work.
%%
%%A “Standard Interface” means an interface that either is an official standard defined by a recognized standards body, or, in the case of interfaces specified for a particular programming language, one that is widely used among developers working in that language.
%%
%%The “System Libraries” of an executable work include anything, other than the work as a whole, that (a) is included in the normal form of packaging a Major Component, but which is not part of that Major Component, and (b) serves only to enable use of the work with that Major Component, or to implement a Standard Interface for which an implementation is available to the public in source code form. A “Major Component”, in this context, means a major essential component (kernel, window system, and so on) of the specific operating system (if any) on which the executable work runs, or a compiler used to produce the work, or an object code interpreter used to run it.
%%
%%The “Corresponding Source” for a work in object code form means all the source code needed to generate, install, and (for an executable work) run the object code and to modify the work, including scripts to control those activities. However, it does not include the work's System Libraries, or general-purpose tools or generally available free programs which are used unmodified in performing those activities but which are not part of the work. For example, Corresponding Source includes interface definition files associated with source files for the work, and the source code for shared libraries and dynamically linked subprograms that the work is specifically designed to require, such as by intimate data communication or control flow between those subprograms and other parts of the work.
%%
%%The Corresponding Source need not include anything that users can regenerate automatically from other parts of the Corresponding Source.
%%
%%The Corresponding Source for a work in source code form is that same work.
%%2. Basic Permissions.
%%
%%All rights granted under this License are granted for the term of copyright on the Program, and are irrevocable provided the stated conditions are met. This License explicitly affirms your unlimited permission to run the unmodified Program. The output from running a covered work is covered by this License only if the output, given its content, constitutes a covered work. This License acknowledges your rights of fair use or other equivalent, as provided by copyright law.
%%
%%You may make, run and propagate covered works that you do not convey, without conditions so long as your license otherwise remains in force. You may convey covered works to others for the sole purpose of having them make modifications exclusively for you, or provide you with facilities for running those works, provided that you comply with the terms of this License in conveying all material for which you do not control copyright. Those thus making or running the covered works for you must do so exclusively on your behalf, under your direction and control, on terms that prohibit them from making any copies of your copyrighted material outside their relationship with you.
%%
%%Conveying under any other circumstances is permitted solely under the conditions stated below. Sublicensing is not allowed; section 10 makes it unnecessary.
%%3. Protecting Users' Legal Rights From Anti-Circumvention Law.
%%
%%No covered work shall be deemed part of an effective technological measure under any applicable law fulfilling obligations under article 11 of the WIPO copyright treaty adopted on 20 December 1996, or similar laws prohibiting or restricting circumvention of such measures.
%%
%%When you convey a covered work, you waive any legal power to forbid circumvention of technological measures to the extent such circumvention is effected by exercising rights under this License with respect to the covered work, and you disclaim any intention to limit operation or modification of the work as a means of enforcing, against the work's users, your or third parties' legal rights to forbid circumvention of technological measures.
%%4. Conveying Verbatim Copies.
%%
%%You may convey verbatim copies of the Program's source code as you receive it, in any medium, provided that you conspicuously and appropriately publish on each copy an appropriate copyright notice; keep intact all notices stating that this License and any non-permissive terms added in accord with section 7 apply to the code; keep intact all notices of the absence of any warranty; and give all recipients a copy of this License along with the Program.
%%
%%You may charge any price or no price for each copy that you convey, and you may offer support or warranty protection for a fee.
%%5. Conveying Modified Source Versions.
%%
%%You may convey a work based on the Program, or the modifications to produce it from the Program, in the form of source code under the terms of section 4, provided that you also meet all of these conditions:
%%
%%    * a) The work must carry prominent notices stating that you modified it, and giving a relevant date.
%%    * b) The work must carry prominent notices stating that it is released under this License and any conditions added under section 7. This requirement modifies the requirement in section 4 to “keep intact all notices”.
%%    * c) You must license the entire work, as a whole, under this License to anyone who comes into possession of a copy. This License will therefore apply, along with any applicable section 7 additional terms, to the whole of the work, and all its parts, regardless of how they are packaged. This License gives no permission to license the work in any other way, but it does not invalidate such permission if you have separately received it.
%%    * d) If the work has interactive user interfaces, each must display Appropriate Legal Notices; however, if the Program has interactive interfaces that do not display Appropriate Legal Notices, your work need not make them do so.
%%
%%A compilation of a covered work with other separate and independent works, which are not by their nature extensions of the covered work, and which are not combined with it such as to form a larger program, in or on a volume of a storage or distribution medium, is called an “aggregate” if the compilation and its resulting copyright are not used to limit the access or legal rights of the compilation's users beyond what the individual works permit. Inclusion of a covered work in an aggregate does not cause this License to apply to the other parts of the aggregate.
%%6. Conveying Non-Source Forms.
%%
%%You may convey a covered work in object code form under the terms of sections 4 and 5, provided that you also convey the machine-readable Corresponding Source under the terms of this License, in one of these ways:
%%
%%    * a) Convey the object code in, or embodied in, a physical product (including a physical distribution medium), accompanied by the Corresponding Source fixed on a durable physical medium customarily used for software interchange.
%%    * b) Convey the object code in, or embodied in, a physical product (including a physical distribution medium), accompanied by a written offer, valid for at least three years and valid for as long as you offer spare parts or customer support for that product model, to give anyone who possesses the object code either (1) a copy of the Corresponding Source for all the software in the product that is covered by this License, on a durable physical medium customarily used for software interchange, for a price no more than your reasonable cost of physically performing this conveying of source, or (2) access to copy the Corresponding Source from a network server at no charge.
%%    * c) Convey individual copies of the object code with a copy of the written offer to provide the Corresponding Source. This alternative is allowed only occasionally and noncommercially, and only if you received the object code with such an offer, in accord with subsection 6b.
%%    * d) Convey the object code by offering access from a designated place (gratis or for a charge), and offer equivalent access to the Corresponding Source in the same way through the same place at no further charge. You need not require recipients to copy the Corresponding Source along with the object code. If the place to copy the object code is a network server, the Corresponding Source may be on a different server (operated by you or a third party) that supports equivalent copying facilities, provided you maintain clear directions next to the object code saying where to find the Corresponding Source. Regardless of what server hosts the Corresponding Source, you remain obligated to ensure that it is available for as long as needed to satisfy these requirements.
%%    * e) Convey the object code using peer-to-peer transmission, provided you inform other peers where the object code and Corresponding Source of the work are being offered to the general public at no charge under subsection 6d.
%%
%%A separable portion of the object code, whose source code is excluded from the Corresponding Source as a System Library, need not be included in conveying the object code work.
%%
%%A “User Product” is either (1) a “consumer product”, which means any tangible personal property which is normally used for personal, family, or household purposes, or (2) anything designed or sold for incorporation into a dwelling. In determining whether a product is a consumer product, doubtful cases shall be resolved in favor of coverage. For a particular product received by a particular user, “normally used” refers to a typical or common use of that class of product, regardless of the status of the particular user or of the way in which the particular user actually uses, or expects or is expected to use, the product. A product is a consumer product regardless of whether the product has substantial commercial, industrial or non-consumer uses, unless such uses represent the only significant mode of use of the product.
%%
%%“Installation Information” for a User Product means any methods, procedures, authorization keys, or other information required to install and execute modified versions of a covered work in that User Product from a modified version of its Corresponding Source. The information must suffice to ensure that the continued functioning of the modified object code is in no case prevented or interfered with solely because modification has been made.
%%
%%If you convey an object code work under this section in, or with, or specifically for use in, a User Product, and the conveying occurs as part of a transaction in which the right of possession and use of the User Product is transferred to the recipient in perpetuity or for a fixed term (regardless of how the transaction is characterized), the Corresponding Source conveyed under this section must be accompanied by the Installation Information. But this requirement does not apply if neither you nor any third party retains the ability to install modified object code on the User Product (for example, the work has been installed in ROM).
%%
%%The requirement to provide Installation Information does not include a requirement to continue to provide support service, warranty, or updates for a work that has been modified or installed by the recipient, or for the User Product in which it has been modified or installed. Access to a network may be denied when the modification itself materially and adversely affects the operation of the network or violates the rules and protocols for communication across the network.
%%
%%Corresponding Source conveyed, and Installation Information provided, in accord with this section must be in a format that is publicly documented (and with an implementation available to the public in source code form), and must require no special password or key for unpacking, reading or copying.
%%7. Additional Terms.
%%
%%“Additional permissions” are terms that supplement the terms of this License by making exceptions from one or more of its conditions. Additional permissions that are applicable to the entire Program shall be treated as though they were included in this License, to the extent that they are valid under applicable law. If additional permissions apply only to part of the Program, that part may be used separately under those permissions, but the entire Program remains governed by this License without regard to the additional permissions.
%%
%%When you convey a copy of a covered work, you may at your option remove any additional permissions from that copy, or from any part of it. (Additional permissions may be written to require their own removal in certain cases when you modify the work.) You may place additional permissions on material, added by you to a covered work, for which you have or can give appropriate copyright permission.
%%
%%Notwithstanding any other provision of this License, for material you add to a covered work, you may (if authorized by the copyright holders of that material) supplement the terms of this License with terms:
%%
%%    * a) Disclaiming warranty or limiting liability differently from the terms of sections 15 and 16 of this License; or
%%    * b) Requiring preservation of specified reasonable legal notices or author attributions in that material or in the Appropriate Legal Notices displayed by works containing it; or
%%    * c) Prohibiting misrepresentation of the origin of that material, or requiring that modified versions of such material be marked in reasonable ways as different from the original version; or
%%    * d) Limiting the use for publicity purposes of names of licensors or authors of the material; or
%%    * e) Declining to grant rights under trademark law for use of some trade names, trademarks, or service marks; or
%%    * f) Requiring indemnification of licensors and authors of that material by anyone who conveys the material (or modified versions of it) with contractual assumptions of liability to the recipient, for any liability that these contractual assumptions directly impose on those licensors and authors.
%%
%%All other non-permissive additional terms are considered “further restrictions” within the meaning of section 10. If the Program as you received it, or any part of it, contains a notice stating that it is governed by this License along with a term that is a further restriction, you may remove that term. If a license document contains a further restriction but permits relicensing or conveying under this License, you may add to a covered work material governed by the terms of that license document, provided that the further restriction does not survive such relicensing or conveying.
%%
%%If you add terms to a covered work in accord with this section, you must place, in the relevant source files, a statement of the additional terms that apply to those files, or a notice indicating where to find the applicable terms.
%%
%%Additional terms, permissive or non-permissive, may be stated in the form of a separately written license, or stated as exceptions; the above requirements apply either way.
%%8. Termination.
%%
%%You may not propagate or modify a covered work except as expressly provided under this License. Any attempt otherwise to propagate or modify it is void, and will automatically terminate your rights under this License (including any patent licenses granted under the third paragraph of section 11).
%%
%%However, if you cease all violation of this License, then your license from a particular copyright holder is reinstated (a) provisionally, unless and until the copyright holder explicitly and finally terminates your license, and (b) permanently, if the copyright holder fails to notify you of the violation by some reasonable means prior to 60 days after the cessation.
%%
%%Moreover, your license from a particular copyright holder is reinstated permanently if the copyright holder notifies you of the violation by some reasonable means, this is the first time you have received notice of violation of this License (for any work) from that copyright holder, and you cure the violation prior to 30 days after your receipt of the notice.
%%
%%Termination of your rights under this section does not terminate the licenses of parties who have received copies or rights from you under this License. If your rights have been terminated and not permanently reinstated, you do not qualify to receive new licenses for the same material under section 10.
%%9. Acceptance Not Required for Having Copies.
%%
%%You are not required to accept this License in order to receive or run a copy of the Program. Ancillary propagation of a covered work occurring solely as a consequence of using peer-to-peer transmission to receive a copy likewise does not require acceptance. However, nothing other than this License grants you permission to propagate or modify any covered work. These actions infringe copyright if you do not accept this License. Therefore, by modifying or propagating a covered work, you indicate your acceptance of this License to do so.
%%10. Automatic Licensing of Downstream Recipients.
%%
%%Each time you convey a covered work, the recipient automatically receives a license from the original licensors, to run, modify and propagate that work, subject to this License. You are not responsible for enforcing compliance by third parties with this License.
%%
%%An “entity transaction” is a transaction transferring control of an organization, or substantially all assets of one, or subdividing an organization, or merging organizations. If propagation of a covered work results from an entity transaction, each party to that transaction who receives a copy of the work also receives whatever licenses to the work the party's predecessor in interest had or could give under the previous paragraph, plus a right to possession of the Corresponding Source of the work from the predecessor in interest, if the predecessor has it or can get it with reasonable efforts.
%%
%%You may not impose any further restrictions on the exercise of the rights granted or affirmed under this License. For example, you may not impose a license fee, royalty, or other charge for exercise of rights granted under this License, and you may not initiate litigation (including a cross-claim or counterclaim in a lawsuit) alleging that any patent claim is infringed by making, using, selling, offering for sale, or importing the Program or any portion of it.
%%11. Patents.
%%
%%A “contributor” is a copyright holder who authorizes use under this License of the Program or a work on which the Program is based. The work thus licensed is called the contributor's “contributor version”.
%%
%%A contributor's “essential patent claims” are all patent claims owned or controlled by the contributor, whether already acquired or hereafter acquired, that would be infringed by some manner, permitted by this License, of making, using, or selling its contributor version, but do not include claims that would be infringed only as a consequence of further modification of the contributor version. For purposes of this definition, “control” includes the right to grant patent sublicenses in a manner consistent with the requirements of this License.
%%
%%Each contributor grants you a non-exclusive, worldwide, royalty-free patent license under the contributor's essential patent claims, to make, use, sell, offer for sale, import and otherwise run, modify and propagate the contents of its contributor version.
%%
%%In the following three paragraphs, a “patent license” is any express agreement or commitment, however denominated, not to enforce a patent (such as an express permission to practice a patent or covenant not to sue for patent infringement). To “grant” such a patent license to a party means to make such an agreement or commitment not to enforce a patent against the party.
%%
%%If you convey a covered work, knowingly relying on a patent license, and the Corresponding Source of the work is not available for anyone to copy, free of charge and under the terms of this License, through a publicly available network server or other readily accessible means, then you must either (1) cause the Corresponding Source to be so available, or (2) arrange to deprive yourself of the benefit of the patent license for this particular work, or (3) arrange, in a manner consistent with the requirements of this License, to extend the patent license to downstream recipients. “Knowingly relying” means you have actual knowledge that, but for the patent license, your conveying the covered work in a country, or your recipient's use of the covered work in a country, would infringe one or more identifiable patents in that country that you have reason to believe are valid.
%%
%%If, pursuant to or in connection with a single transaction or arrangement, you convey, or propagate by procuring conveyance of, a covered work, and grant a patent license to some of the parties receiving the covered work authorizing them to use, propagate, modify or convey a specific copy of the covered work, then the patent license you grant is automatically extended to all recipients of the covered work and works based on it.
%%
%%A patent license is “discriminatory” if it does not include within the scope of its coverage, prohibits the exercise of, or is conditioned on the non-exercise of one or more of the rights that are specifically granted under this License. You may not convey a covered work if you are a party to an arrangement with a third party that is in the business of distributing software, under which you make payment to the third party based on the extent of your activity of conveying the work, and under which the third party grants, to any of the parties who would receive the covered work from you, a discriminatory patent license (a) in connection with copies of the covered work conveyed by you (or copies made from those copies), or (b) primarily for and in connection with specific products or compilations that contain the covered work, unless you entered into that arrangement, or that patent license was granted, prior to 28 March 2007.
%%
%%Nothing in this License shall be construed as excluding or limiting any implied license or other defenses to infringement that may otherwise be available to you under applicable patent law.
%%12. No Surrender of Others' Freedom.
%%
%%If conditions are imposed on you (whether by court order, agreement or otherwise) that contradict the conditions of this License, they do not excuse you from the conditions of this License. If you cannot convey a covered work so as to satisfy simultaneously your obligations under this License and any other pertinent obligations, then as a consequence you may not convey it at all. For example, if you agree to terms that obligate you to collect a royalty for further conveying from those to whom you convey the Program, the only way you could satisfy both those terms and this License would be to refrain entirely from conveying the Program.
%%13. Use with the GNU Affero General Public License.
%%
%%Notwithstanding any other provision of this License, you have permission to link or combine any covered work with a work licensed under version 3 of the GNU Affero General Public License into a single combined work, and to convey the resulting work. The terms of this License will continue to apply to the part which is the covered work, but the special requirements of the GNU Affero General Public License, section 13, concerning interaction through a network will apply to the combination as such.
%%14. Revised Versions of this License.
%%
%%The Free Software Foundation may publish revised and/or new versions of the GNU General Public License from time to time. Such new versions will be similar in spirit to the present version, but may differ in detail to address new problems or concerns.
%%
%%Each version is given a distinguishing version number. If the Program specifies that a certain numbered version of the GNU General Public License “or any later version” applies to it, you have the option of following the terms and conditions either of that numbered version or of any later version published by the Free Software Foundation. If the Program does not specify a version number of the GNU General Public License, you may choose any version ever published by the Free Software Foundation.
%%
%%If the Program specifies that a proxy can decide which future versions of the GNU General Public License can be used, that proxy's public statement of acceptance of a version permanently authorizes you to choose that version for the Program.
%%
%%Later license versions may give you additional or different permissions. However, no additional obligations are imposed on any author or copyright holder as a result of your choosing to follow a later version.
%%15. Disclaimer of Warranty.
%%
%%THERE IS NO WARRANTY FOR THE PROGRAM, TO THE EXTENT PERMITTED BY APPLICABLE LAW. EXCEPT WHEN OTHERWISE STATED IN WRITING THE COPYRIGHT HOLDERS AND/OR OTHER PARTIES PROVIDE THE PROGRAM “AS IS” WITHOUT WARRANTY OF ANY KIND, EITHER EXPRESSED OR IMPLIED, INCLUDING, BUT NOT LIMITED TO, THE IMPLIED WARRANTIES OF MERCHANTABILITY AND FITNESS FOR A PARTICULAR PURPOSE. THE ENTIRE RISK AS TO THE QUALITY AND PERFORMANCE OF THE PROGRAM IS WITH YOU. SHOULD THE PROGRAM PROVE DEFECTIVE, YOU ASSUME THE COST OF ALL NECESSARY SERVICING, REPAIR OR CORRECTION.
%%16. Limitation of Liability.
%%
%%IN NO EVENT UNLESS REQUIRED BY APPLICABLE LAW OR AGREED TO IN WRITING WILL ANY COPYRIGHT HOLDER, OR ANY OTHER PARTY WHO MODIFIES AND/OR CONVEYS THE PROGRAM AS PERMITTED ABOVE, BE LIABLE TO YOU FOR DAMAGES, INCLUDING ANY GENERAL, SPECIAL, INCIDENTAL OR CONSEQUENTIAL DAMAGES ARISING OUT OF THE USE OR INABILITY TO USE THE PROGRAM (INCLUDING BUT NOT LIMITED TO LOSS OF DATA OR DATA BEING RENDERED INACCURATE OR LOSSES SUSTAINED BY YOU OR THIRD PARTIES OR A FAILURE OF THE PROGRAM TO OPERATE WITH ANY OTHER PROGRAMS), EVEN IF SUCH HOLDER OR OTHER PARTY HAS BEEN ADVISED OF THE POSSIBILITY OF SUCH DAMAGES.
%%17. Interpretation of Sections 15 and 16.
%%
%%If the disclaimer of warranty and limitation of liability provided above cannot be given local legal effect according to their terms, reviewing courts shall apply local law that most closely approximates an absolute waiver of all civil liability in connection with the Program, unless a warranty or assumption of liability accompanies a copy of the Program in return for a fee.
%%
%%END OF TERMS AND CONDITIONS
%%How to Apply These Terms to Your New Programs
%%
%%If you develop a new program, and you want it to be of the greatest possible use to the public, the best way to achieve this is to make it free software which everyone can redistribute and change under these terms.
%%
%%To do so, attach the following notices to the program. It is safest to attach them to the start of each source file to most effectively state the exclusion of warranty; and each file should have at least the “copyright” line and a pointer to where the full notice is found.
%%
%%    <one line to give the program's name and a brief idea of what it does.>
%%    Copyright (C) <year>  <name of author>
%%
%%    This program is free software: you can redistribute it and/or modify
%%    it under the terms of the GNU General Public License as published by
%%    the Free Software Foundation, either version 3 of the License, or
%%    (at your option) any later version.
%%
%%    This program is distributed in the hope that it will be useful,
%%    but WITHOUT ANY WARRANTY; without even the implied warranty of
%%    MERCHANTABILITY or FITNESS FOR A PARTICULAR PURPOSE.  See the
%%    GNU General Public License for more details.
%%
%%    You should have received a copy of the GNU General Public License
%%    along with this program.  If not, see <http://www.gnu.org/licenses/>.
%%
%%Also add information on how to contact you by electronic and paper mail.
%%
%%If the program does terminal interaction, make it output a short notice like this when it starts in an interactive mode:
%%
%%    <program>  Copyright (C) <year>  <name of author>
%%    This program comes with ABSOLUTELY NO WARRANTY; for details type `show w'.
%%    This is free software, and you are welcome to redistribute it
%%    under certain conditions; type `show c' for details.
%%
%%The hypothetical commands `show w' and `show c' should show the appropriate parts of the General Public License. Of course, your program's commands might be different; for a GUI interface, you would use an “about box”.
%%
%%You should also get your employer (if you work as a programmer) or school, if any, to sign a “copyright disclaimer” for the program, if necessary. For more information on this, and how to apply and follow the GNU GPL, see <http://www.gnu.org/licenses/>.
%%
%%The GNU General Public License does not permit incorporating your program into proprietary programs. If your program is a subroutine library, you may consider it more useful to permit linking proprietary applications with the library. If this is what you want to do, use the GNU Lesser General Public License instead of this License. But first, please read <http://www.gnu.org/philosophy/why-not-lgpl.html>.
%%\end{multicols}
%%
%%\section{GNU Free Documentation License}
%%\begin{multicols}{4}
%%
%%Version 1.3, 3 November 2008
%%
%%Copyright © 2000, 2001, 2002, 2007, 2008 Free Software Foundation, Inc. <http://fsf.org/>
%%
%%Everyone is permitted to copy and distribute verbatim copies of this license document, but changing it is not allowed.
%%0. PREAMBLE
%%
%%The purpose of this License is to make a manual, textbook, or other functional and useful document "free" in the sense of freedom: to assure everyone the effective freedom to copy and redistribute it, with or without modifying it, either commercially or noncommercially. Secondarily, this License preserves for the author and publisher a way to get credit for their work, while not being considered responsible for modifications made by others.
%%
%%This License is a kind of "copyleft", which means that derivative works of the document must themselves be free in the same sense. It complements the GNU General Public License, which is a copyleft license designed for free software.
%%
%%We have designed this License in order to use it for manuals for free software, because free software needs free documentation: a free program should come with manuals providing the same freedoms that the software does. But this License is not limited to software manuals; it can be used for any textual work, regardless of subject matter or whether it is published as a printed book. We recommend this License principally for works whose purpose is instruction or reference.
%%1. APPLICABILITY AND DEFINITIONS
%%
%%This License applies to any manual or other work, in any medium, that contains a notice placed by the copyright holder saying it can be distributed under the terms of this License. Such a notice grants a world-wide, royalty-free license, unlimited in duration, to use that work under the conditions stated herein. The "Document", below, refers to any such manual or work. Any member of the public is a licensee, and is addressed as "you". You accept the license if you copy, modify or distribute the work in a way requiring permission under copyright law.
%%
%%A "Modified Version" of the Document means any work containing the Document or a portion of it, either copied verbatim, or with modifications and/or translated into another language.
%%
%%A "Secondary Section" is a named appendix or a front-matter section of the Document that deals exclusively with the relationship of the publishers or authors of the Document to the Document's overall subject (or to related matters) and contains nothing that could fall directly within that overall subject. (Thus, if the Document is in part a textbook of mathematics, a Secondary Section may not explain any mathematics.) The relationship could be a matter of historical connection with the subject or with related matters, or of legal, commercial, philosophical, ethical or political position regarding them.
%%
%%The "Invariant Sections" are certain Secondary Sections whose titles are designated, as being those of Invariant Sections, in the notice that says that the Document is released under this License. If a section does not fit the above definition of Secondary then it is not allowed to be designated as Invariant. The Document may contain zero Invariant Sections. If the Document does not identify any Invariant Sections then there are none.
%%
%%The "Cover Texts" are certain short passages of text that are listed, as Front-Cover Texts or Back-Cover Texts, in the notice that says that the Document is released under this License. A Front-Cover Text may be at most 5 words, and a Back-Cover Text may be at most 25 words.
%%
%%A "Transparent" copy of the Document means a machine-readable copy, represented in a format whose specification is available to the general public, that is suitable for revising the document straightforwardly with generic text editors or (for images composed of pixels) generic paint programs or (for drawings) some widely available drawing editor, and that is suitable for input to text formatters or for automatic translation to a variety of formats suitable for input to text formatters. A copy made in an otherwise Transparent file format whose markup, or absence of markup, has been arranged to thwart or discourage subsequent modification by readers is not Transparent. An image format is not Transparent if used for any substantial amount of text. A copy that is not "Transparent" is called "Opaque".
%%
%%Examples of suitable formats for Transparent copies include plain ASCII without markup, Texinfo input format, LaTeX input format, SGML or XML using a publicly available DTD, and standard-conforming simple HTML, PostScript or PDF designed for human modification. Examples of transparent image formats include PNG, XCF and JPG. Opaque formats include proprietary formats that can be read and edited only by proprietary word processors, SGML or XML for which the DTD and/or processing tools are not generally available, and the machine-generated HTML, PostScript or PDF produced by some word processors for output purposes only.
%%
%%The "Title Page" means, for a printed book, the title page itself, plus such following pages as are needed to hold, legibly, the material this License requires to appear in the title page. For works in formats which do not have any title page as such, "Title Page" means the text near the most prominent appearance of the work's title, preceding the beginning of the body of the text.
%%
%%The "publisher" means any person or entity that distributes copies of the Document to the public.
%%
%%A section "Entitled XYZ" means a named subunit of the Document whose title either is precisely XYZ or contains XYZ in parentheses following text that translates XYZ in another language. (Here XYZ stands for a specific section name mentioned below, such as "Acknowledgements", "Dedications", "Endorsements", or "History".) To "Preserve the Title" of such a section when you modify the Document means that it remains a section "Entitled XYZ" according to this definition.
%%
%%The Document may include Warranty Disclaimers next to the notice which states that this License applies to the Document. These Warranty Disclaimers are considered to be included by reference in this License, but only as regards disclaiming warranties: any other implication that these Warranty Disclaimers may have is void and has no effect on the meaning of this License.
%%2. VERBATIM COPYING
%%
%%You may copy and distribute the Document in any medium, either commercially or noncommercially, provided that this License, the copyright notices, and the license notice saying this License applies to the Document are reproduced in all copies, and that you add no other conditions whatsoever to those of this License. You may not use technical measures to obstruct or control the reading or further copying of the copies you make or distribute. However, you may accept compensation in exchange for copies. If you distribute a large enough number of copies you must also follow the conditions in section 3.
%%
%%You may also lend copies, under the same conditions stated above, and you may publicly display copies.
%%3. COPYING IN QUANTITY
%%
%%If you publish printed copies (or copies in media that commonly have printed covers) of the Document, numbering more than 100, and the Document's license notice requires Cover Texts, you must enclose the copies in covers that carry, clearly and legibly, all these Cover Texts: Front-Cover Texts on the front cover, and Back-Cover Texts on the back cover. Both covers must also clearly and legibly identify you as the publisher of these copies. The front cover must present the full title with all words of the title equally prominent and visible. You may add other material on the covers in addition. Copying with changes limited to the covers, as long as they preserve the title of the Document and satisfy these conditions, can be treated as verbatim copying in other respects.
%%
%%If the required texts for either cover are too voluminous to fit legibly, you should put the first ones listed (as many as fit reasonably) on the actual cover, and continue the rest onto adjacent pages.
%%
%%If you publish or distribute Opaque copies of the Document numbering more than 100, you must either include a machine-readable Transparent copy along with each Opaque copy, or state in or with each Opaque copy a computer-network location from which the general network-using public has access to download using public-standard network protocols a complete Transparent copy of the Document, free of added material. If you use the latter option, you must take reasonably prudent steps, when you begin distribution of Opaque copies in quantity, to ensure that this Transparent copy will remain thus accessible at the stated location until at least one year after the last time you distribute an Opaque copy (directly or through your agents or retailers) of that edition to the public.
%%
%%It is requested, but not required, that you contact the authors of the Document well before redistributing any large number of copies, to give them a chance to provide you with an updated version of the Document.
%%4. MODIFICATIONS
%%
%%You may copy and distribute a Modified Version of the Document under the conditions of sections 2 and 3 above, provided that you release the Modified Version under precisely this License, with the Modified Version filling the role of the Document, thus licensing distribution and modification of the Modified Version to whoever possesses a copy of it. In addition, you must do these things in the Modified Version:
%%
%%    * A. Use in the Title Page (and on the covers, if any) a title distinct from that of the Document, and from those of previous versions (which should, if there were any, be listed in the History section of the Document). You may use the same title as a previous version if the original publisher of that version gives permission.
%%    * B. List on the Title Page, as authors, one or more persons or entities responsible for authorship of the modifications in the Modified Version, together with at least five of the principal authors of the Document (all of its principal authors, if it has fewer than five), unless they release you from this requirement.
%%    * C. State on the Title page the name of the publisher of the Modified Version, as the publisher.
%%    * D. Preserve all the copyright notices of the Document.
%%    * E. Add an appropriate copyright notice for your modifications adjacent to the other copyright notices.
%%    * F. Include, immediately after the copyright notices, a license notice giving the public permission to use the Modified Version under the terms of this License, in the form shown in the Addendum below.
%%    * G. Preserve in that license notice the full lists of Invariant Sections and required Cover Texts given in the Document's license notice.
%%    * H. Include an unaltered copy of this License.
%%    * I. Preserve the section Entitled "History", Preserve its Title, and add to it an item stating at least the title, year, new authors, and publisher of the Modified Version as given on the Title Page. If there is no section Entitled "History" in the Document, create one stating the title, year, authors, and publisher of the Document as given on its Title Page, then add an item describing the Modified Version as stated in the previous sentence.
%%    * J. Preserve the network location, if any, given in the Document for public access to a Transparent copy of the Document, and likewise the network locations given in the Document for previous versions it was based on. These may be placed in the "History" section. You may omit a network location for a work that was published at least four years before the Document itself, or if the original publisher of the version it refers to gives permission.
%%    * K. For any section Entitled "Acknowledgements" or "Dedications", Preserve the Title of the section, and preserve in the section all the substance and tone of each of the contributor acknowledgements and/or dedications given therein.
%%    * L. Preserve all the Invariant Sections of the Document, unaltered in their text and in their titles. Section numbers or the equivalent are not considered part of the section titles.
%%    * M. Delete any section Entitled "Endorsements". Such a section may not be included in the Modified Version.
%%    * N. Do not retitle any existing section to be Entitled "Endorsements" or to conflict in title with any Invariant Section.
%%    * O. Preserve any Warranty Disclaimers.
%%
%%If the Modified Version includes new front-matter sections or appendices that qualify as Secondary Sections and contain no material copied from the Document, you may at your option designate some or all of these sections as invariant. To do this, add their titles to the list of Invariant Sections in the Modified Version's license notice. These titles must be distinct from any other section titles.
%%
%%You may add a section Entitled "Endorsements", provided it contains nothing but endorsements of your Modified Version by various parties—for example, statements of peer review or that the text has been approved by an organization as the authoritative definition of a standard.
%%
%%You may add a passage of up to five words as a Front-Cover Text, and a passage of up to 25 words as a Back-Cover Text, to the end of the list of Cover Texts in the Modified Version. Only one passage of Front-Cover Text and one of Back-Cover Text may be added by (or through arrangements made by) any one entity. If the Document already includes a cover text for the same cover, previously added by you or by arrangement made by the same entity you are acting on behalf of, you may not add another; but you may replace the old one, on explicit permission from the previous publisher that added the old one.
%%
%%The author(s) and publisher(s) of the Document do not by this License give permission to use their names for publicity for or to assert or imply endorsement of any Modified Version.
%%5. COMBINING DOCUMENTS
%%
%%You may combine the Document with other documents released under this License, under the terms defined in section 4 above for modified versions, provided that you include in the combination all of the Invariant Sections of all of the original documents, unmodified, and list them all as Invariant Sections of your combined work in its license notice, and that you preserve all their Warranty Disclaimers.
%%
%%The combined work need only contain one copy of this License, and multiple identical Invariant Sections may be replaced with a single copy. If there are multiple Invariant Sections with the same name but different contents, make the title of each such section unique by adding at the end of it, in parentheses, the name of the original author or publisher of that section if known, or else a unique number. Make the same adjustment to the section titles in the list of Invariant Sections in the license notice of the combined work.
%%
%%In the combination, you must combine any sections Entitled "History" in the various original documents, forming one section Entitled "History"; likewise combine any sections Entitled "Acknowledgements", and any sections Entitled "Dedications". You must delete all sections Entitled "Endorsements".
%%6. COLLECTIONS OF DOCUMENTS
%%
%%You may make a collection consisting of the Document and other documents released under this License, and replace the individual copies of this License in the various documents with a single copy that is included in the collection, provided that you follow the rules of this License for verbatim copying of each of the documents in all other respects.
%%
%%You may extract a single document from such a collection, and distribute it individually under this License, provided you insert a copy of this License into the extracted document, and follow this License in all other respects regarding verbatim copying of that document.
%%7. AGGREGATION WITH INDEPENDENT WORKS
%%
%%A compilation of the Document or its derivatives with other separate and independent documents or works, in or on a volume of a storage or distribution medium, is called an "aggregate" if the copyright resulting from the compilation is not used to limit the legal rights of the compilation's users beyond what the individual works permit. When the Document is included in an aggregate, this License does not apply to the other works in the aggregate which are not themselves derivative works of the Document.
%%
%%If the Cover Text requirement of section 3 is applicable to these copies of the Document, then if the Document is less than one half of the entire aggregate, the Document's Cover Texts may be placed on covers that bracket the Document within the aggregate, or the electronic equivalent of covers if the Document is in electronic form. Otherwise they must appear on printed covers that bracket the whole aggregate.
%%8. TRANSLATION
%%
%%Translation is considered a kind of modification, so you may distribute translations of the Document under the terms of section 4. Replacing Invariant Sections with translations requires special permission from their copyright holders, but you may include translations of some or all Invariant Sections in addition to the original versions of these Invariant Sections. You may include a translation of this License, and all the license notices in the Document, and any Warranty Disclaimers, provided that you also include the original English version of this License and the original versions of those notices and disclaimers. In case of a disagreement between the translation and the original version of this License or a notice or disclaimer, the original version will prevail.
%%
%%If a section in the Document is Entitled "Acknowledgements", "Dedications", or "History", the requirement (section 4) to Preserve its Title (section 1) will typically require changing the actual title.
%%9. TERMINATION
%%
%%You may not copy, modify, sublicense, or distribute the Document except as expressly provided under this License. Any attempt otherwise to copy, modify, sublicense, or distribute it is void, and will automatically terminate your rights under this License.
%%
%%However, if you cease all violation of this License, then your license from a particular copyright holder is reinstated (a) provisionally, unless and until the copyright holder explicitly and finally terminates your license, and (b) permanently, if the copyright holder fails to notify you of the violation by some reasonable means prior to 60 days after the cessation.
%%
%%Moreover, your license from a particular copyright holder is reinstated permanently if the copyright holder notifies you of the violation by some reasonable means, this is the first time you have received notice of violation of this License (for any work) from that copyright holder, and you cure the violation prior to 30 days after your receipt of the notice.
%%
%%Termination of your rights under this section does not terminate the licenses of parties who have received copies or rights from you under this License. If your rights have been terminated and not permanently reinstated, receipt of a copy of some or all of the same material does not give you any rights to use it.
%%10. FUTURE REVISIONS OF THIS LICENSE
%%
%%The Free Software Foundation may publish new, revised versions of the GNU Free Documentation License from time to time. Such new versions will be similar in spirit to the present version, but may differ in detail to address new problems or concerns. See http://www.gnu.org/copyleft/.
%%
%%Each version of the License is given a distinguishing version number. If the Document specifies that a particular numbered version of this License "or any later version" applies to it, you have the option of following the terms and conditions either of that specified version or of any later version that has been published (not as a draft) by the Free Software Foundation. If the Document does not specify a version number of this License, you may choose any version ever published (not as a draft) by the Free Software Foundation. If the Document specifies that a proxy can decide which future versions of this License can be used, that proxy's public statement of acceptance of a version permanently authorizes you to choose that version for the Document.
%%11. RELICENSING
%%
%%"Massive Multiauthor Collaboration Site" (or "MMC Site") means any World Wide Web server that publishes copyrightable works and also provides prominent facilities for anybody to edit those works. A public wiki that anybody can edit is an example of such a server. A "Massive Multiauthor Collaboration" (or "MMC") contained in the site means any set of copyrightable works thus published on the MMC site.
%%
%%"CC-BY-SA" means the Creative Commons Attribution-Share Alike 3.0 license published by Creative Commons Corporation, a not-for-profit corporation with a principal place of business in San Francisco, California, as well as future copyleft versions of that license published by that same organization.
%%
%%"Incorporate" means to publish or republish a Document, in whole or in part, as part of another Document.
%%
%%An MMC is "eligible for relicensing" if it is licensed under this License, and if all works that were first published under this License somewhere other than this MMC, and subsequently incorporated in whole or in part into the MMC, (1) had no cover texts or invariant sections, and (2) were thus incorporated prior to November 1, 2008.
%%
%%The operator of an MMC Site may republish an MMC contained in the site under CC-BY-SA on the same site at any time before August 1, 2009, provided the MMC is eligible for relicensing.
%%ADDENDUM: How to use this License for your documents
%%
%%To use this License in a document you have written, include a copy of the License in the document and put the following copyright and license notices just after the title page:
%%
%%    Copyright (C)  YEAR  YOUR NAME.
%%    Permission is granted to copy, distribute and/or modify this document
%%    under the terms of the GNU Free Documentation License, Version 1.3
%%    or any later version published by the Free Software Foundation;
%%    with no Invariant Sections, no Front-Cover Texts, and no Back-Cover Texts.
%%    A copy of the license is included in the section entitled "GNU
%%    Free Documentation License".
%%
%%If you have Invariant Sections, Front-Cover Texts and Back-Cover Texts, replace the "with … Texts." line with this:
%%
%%    with the Invariant Sections being LIST THEIR TITLES, with the
%%    Front-Cover Texts being LIST, and with the Back-Cover Texts being LIST.
%%
%%If you have Invariant Sections without Cover Texts, or some other combination of the three, merge those two alternatives to suit the situation.
%%
%%If your document contains nontrivial examples of program code, we recommend releasing these examples in parallel under your choice of free software license, such as the GNU General Public License, to permit their use in free software.
%%\end{multicols}
%%
%%\section{GNU Lesser General Public License}
%%\begin{multicols}{4}
%%
%%
%%GNU LESSER GENERAL PUBLIC LICENSE
%%
%%Version 3, 29 June 2007
%%
%%Copyright © 2007 Free Software Foundation, Inc. <http://fsf.org/>
%%
%%Everyone is permitted to copy and distribute verbatim copies of this license document, but changing it is not allowed.
%%
%%This version of the GNU Lesser General Public License incorporates the terms and conditions of version 3 of the GNU General Public License, supplemented by the additional permissions listed below.
%%0. Additional Definitions.
%%
%%As used herein, “this License” refers to version 3 of the GNU Lesser General Public License, and the “GNU GPL” refers to version 3 of the GNU General Public License.
%%
%%“The Library” refers to a covered work governed by this License, other than an Application or a Combined Work as defined below.
%%
%%An “Application” is any work that makes use of an interface provided by the Library, but which is not otherwise based on the Library. Defining a subclass of a class defined by the Library is deemed a mode of using an interface provided by the Library.
%%
%%A “Combined Work” is a work produced by combining or linking an Application with the Library. The particular version of the Library with which the Combined Work was made is also called the “Linked Version”.
%%
%%The “Minimal Corresponding Source” for a Combined Work means the Corresponding Source for the Combined Work, excluding any source code for portions of the Combined Work that, considered in isolation, are based on the Application, and not on the Linked Version.
%%
%%The “Corresponding Application Code” for a Combined Work means the object code and/or source code for the Application, including any data and utility programs needed for reproducing the Combined Work from the Application, but excluding the System Libraries of the Combined Work.
%%1. Exception to Section 3 of the GNU GPL.
%%
%%You may convey a covered work under sections 3 and 4 of this License without being bound by section 3 of the GNU GPL.
%%2. Conveying Modified Versions.
%%
%%If you modify a copy of the Library, and, in your modifications, a facility refers to a function or data to be supplied by an Application that uses the facility (other than as an argument passed when the facility is invoked), then you may convey a copy of the modified version:
%%
%%    * a) under this License, provided that you make a good faith effort to ensure that, in the event an Application does not supply the function or data, the facility still operates, and performs whatever part of its purpose remains meaningful, or
%%    * b) under the GNU GPL, with none of the additional permissions of this License applicable to that copy.
%%
%%3. Object Code Incorporating Material from Library Header Files.
%%
%%The object code form of an Application may incorporate material from a header file that is part of the Library. You may convey such object code under terms of your choice, provided that, if the incorporated material is not limited to numerical parameters, data structure layouts and accessors, or small macros, inline functions and templates (ten or fewer lines in length), you do both of the following:
%%
%%    * a) Give prominent notice with each copy of the object code that the Library is used in it and that the Library and its use are covered by this License.
%%    * b) Accompany the object code with a copy of the GNU GPL and this license document.
%%
%%4. Combined Works.
%%
%%You may convey a Combined Work under terms of your choice that, taken together, effectively do not restrict modification of the portions of the Library contained in the Combined Work and reverse engineering for debugging such modifications, if you also do each of the following:
%%
%%    * a) Give prominent notice with each copy of the Combined Work that the Library is used in it and that the Library and its use are covered by this License.
%%    * b) Accompany the Combined Work with a copy of the GNU GPL and this license document.
%%    * c) For a Combined Work that displays copyright notices during execution, include the copyright notice for the Library among these notices, as well as a reference directing the user to the copies of the GNU GPL and this license document.
%%    * d) Do one of the following:
%%          o 0) Convey the Minimal Corresponding Source under the terms of this License, and the Corresponding Application Code in a form suitable for, and under terms that permit, the user to recombine or relink the Application with a modified version of the Linked Version to produce a modified Combined Work, in the manner specified by section 6 of the GNU GPL for conveying Corresponding Source.
%%          o 1) Use a suitable shared library mechanism for linking with the Library. A suitable mechanism is one that (a) uses at run time a copy of the Library already present on the user's computer system, and (b) will operate properly with a modified version of the Library that is interface-compatible with the Linked Version.
%%    * e) Provide Installation Information, but only if you would otherwise be required to provide such information under section 6 of the GNU GPL, and only to the extent that such information is necessary to install and execute a modified version of the Combined Work produced by recombining or relinking the Application with a modified version of the Linked Version. (If you use option 4d0, the Installation Information must accompany the Minimal Corresponding Source and Corresponding Application Code. If you use option 4d1, you must provide the Installation Information in the manner specified by section 6 of the GNU GPL for conveying Corresponding Source.)
%%
%%5. Combined Libraries.
%%
%%You may place library facilities that are a work based on the Library side by side in a single library together with other library facilities that are not Applications and are not covered by this License, and convey such a combined library under terms of your choice, if you do both of the following:
%%
%%    * a) Accompany the combined library with a copy of the same work based on the Library, uncombined with any other library facilities, conveyed under the terms of this License.
%%    * b) Give prominent notice with the combined library that part of it is a work based on the Library, and explaining where to find the accompanying uncombined form of the same work.
%%
%%6. Revised Versions of the GNU Lesser General Public License.
%%
%%The Free Software Foundation may publish revised and/or new versions of the GNU Lesser General Public License from time to time. Such new versions will be similar in spirit to the present version, but may differ in detail to address new problems or concerns.
%%
%%Each version is given a distinguishing version number. If the Library as you received it specifies that a certain numbered version of the GNU Lesser General Public License “or any later version” applies to it, you have the option of following the terms and conditions either of that published version or of any later version published by the Free Software Foundation. If the Library as you received it does not specify a version number of the GNU Lesser General Public License, you may choose any version of the GNU Lesser General Public License ever published by the Free Software Foundation.
%%
%%If the Library as you received it specifies that a proxy can decide whether future versions of the GNU Lesser General Public License shall apply, that proxy's public statement of acceptance of any version is permanent authorization for you to choose that version for the Library.
%%\end{multicols}
%%}
%%\pagebreak
\end{document}
